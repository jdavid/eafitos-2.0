\chapter{Manual de usuario}

\section{Instalaci�n de Eafitos}
Para poder ejecutar Eafitos tan solo necesitas el fichero ejecutable, el cual
debes haberlo obtenido junto con este documento.

Pero antes de ejecutar Eafitos debes tener en cuenta una cosa.
Eafitos necesita de unos ficheros (los discos duros virtuales) que se
alojaran en el directorio ``$\backslash$EAFITOS'' o ``\$HOME/EAFITOS''
seg�n utilices DOS o GNU/Linux.
Asegurate de tener dicho directorio creado antes de ejecutar Eafitos.

\subsubsection{Compilaci�n}
Para compilar Eafitos en DOS necesitar�s Borland C++ 4.5, es posible
que funcione con otras versiones (posteriores a la versi�n 3.1), pero no
puedo garantizarlo ya que no lo he probado.

Para compilarlo en GNU/Linux necesitar�s el compilador \emph{egcs} y la
librer�a \emph{NCURSES}.
Si dispones de esto, es posible que incluso puedas compilarlo en otras
versiones de UNIX.

La compilaci�n en otros entornos ser� posible siempre y cuando dispongas
de alguna l�breria de Curses y de un compilador de C++ que implemente,
al menos, la revisi�n n�mero 3.0 del borrador oficial de ANSI C++.

\section{Gesti�n de los discos}
La informaci�n en Eafitos se almacena en unos discos virtuales que en realidad
son ficheros del Sistema Operativo real sobre el que se ejecuta Eafitos.

Cuando ejecutas Eafitos aparece un men� con cuatro opciones.
La primera te permitir� arrancar el int�rprete de comandos para empezar a
trabajar, pero antes de hacer esto debes tener al menos un disco creado y
formateado.
Para eso tienes las opciones 2 y 3, primero crea los discos que quieras
(hasta un m�ximo de cinco) con la opci�n 2 y despu�s les das formato;
ahora ya puedes iniciar el int�rprete.
La �ltima opci�n, la cuarta, es, como su nombre indica, para terminar la
ejecuci�n de Eafitos.

\section{El int�rprete de comandos}
Ahora vamos a ver qu� es lo que se puede hacer con Eafitos.
Para ello estudiaremos su int�rprete de comandos.

Primero, has de saber que el int�rprete distingue entre may�sculas y
min�sculas (como UNIX y al contrario que DOS), por lo que `Salir' no es
lo mismo que `salir'.

Ahora veamos la lista de comandos que nos proporciona:
\begin{itemize}
\item \textbf{salir} Termina el int�rprete y regresa al men� inicial.
\item \textbf{disco {\em numero}} Cambia el disco actual al indicado y el
	directorio actual pasa a ser el directorio ra�z del nuevo disco.
\item \textbf{creadir {\em nombre}} Crea el directorio {\em nombre}.
\item \textbf{cd {\em directorio}} Cambia el directorio actual al indicado.
\item \textbf{poner {\em fichero}} Pasa el fichero indicado del SO anfitri�n
	a Eafitos.
\item \textbf{obtener {\em fichero}} Pasa el fichero indicado de Eafitos al SO
	anfitri�n.
\item \textbf{borrar {\em fichero}} Borra el fichero indicado (tambi�n sirve
	para direcotorios).
\item \textbf{dir} Lista el contenido del directorio actual.
\item \textbf{ver {\em fichero}} Muestra el contenido del fichero indicado
	(como \emph{type} de DOS o \emph{cat} de UNIX).
\item \textbf{compilar {\em fichero}} Compila el fichero indicado (aqu�
	entra en juego el compilador), el ejecutable que se genera recibe
	el nombre {\em fichero.exe}.
\end{itemize}

Cuando escribes cualquier otra cadena, si corresponde a un fichero ejecutable
lo ejecuta y si no da un mensaje de error.
Los programas de usuario no admiten par�metros en la l�nea de comandos.
Cualquier par�metro de m�s en la l�nea de comandos se ignora.
