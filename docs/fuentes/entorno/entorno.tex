\chapter{El entorno}

La ejecuci�n comienza en la funci�n \id{main}.
\marginnote{Ver eafitos2.cpp}
Primero inicia la m�quina virtual y el n�cleo, despu�s cede el control al
entorno ejecutando el m�todo \id{Entorno::iniciar} y finalmente termina
el n�cleo y la m�quina virtual.

El entorno no es m�s que el sistema de men�s que aparece cuando se ejecuta
Eafitos, el cual permite crear y formatear discos e iniciar el int�rprete
de comandos.
Est� definido e implementado por la clase \id{Entorno} en los ficheros
\emph{include/entorno.h} y \emph{entorno/entorno.cpp}, es muy simple, as� que
no lo voy a comentar.


\section{El int�rprete de comandos}
El int�rprete de comandos est� definido e implementado por la clase.
\id{InterpreteComandos}.
\marginnote{Ver include/ic.h y entorno/ic.cpp}
Consta de un s�lo m�todo llamado \id{iniciar}, que es muy sencillo.
B�sicamente se trata de un bucle infinito, al principio del cual se lee una
l�nea del teclado para analizarla a continuaci�n.
En funci�n del comando que haya escrito el usuario se realizar� una acci�n u
otra, o ninguna si el comando no est� definido.
Del bucle se sale cuando el usuario escribe el comando \emph{salir}.
