\documentclass[spanish,a4paper]{book}


\usepackage{isolatin1}
\usepackage{babel}
\usepackage[dvips]{epsfig}
\usepackage{moreverb}
\usepackage{personal}

\setlength\parskip{0.8em}
\setlength\parindent{0pt}



\title{EAFITOS\\
	Una herramienta para aprender sistemas operativos\\
	(versi�n 2.0)}
\author{Juan David Ib��ez Palomar}


\begin{document}

\frontmatter

\maketitle

\thispagestyle{empty}
{\Large AVISO LEGAL}

\begin{tabular}{ll}
Copyright \copyright & 1995-1999 Omar Garc�a Palencia\\
& 1998-1999 Juan David Ib��ez Palomar
\end{tabular}

Este documento forma parte del Sistema Operativo Eafitos.

Eafitos es software libre, puedes redistribuirlo y/o modificarlo bajo
los terminos de la Licencia Publica General GNU (GNU General Public
License) publicada por la Fundaci�n del Software Libre (Free Software
Foundation); bien bajo la versi�n 2 de la licencia, o (a tu elecci�n)
cualquier versi�n posterior.

Eafitos es distribuido con la esperanza de que sea �til, pero SIN
NINGUNA GARANTIA. Mira la Licencia Publica General GNU para m�s detalles.

Se adjunta una copia de la Licencia Publica General GNU (en versi�n
original) en el apendice~\ref{GPL} y una traducci�n al castellano en el
apendice~\ref{GPL en castellano}.


\newpage
\thispagestyle{empty}
\vspace*{0.45\textheight}
\hspace{0.5\textwidth}
{\em a mi abuelo Victoriano}


\tableofcontents
\listoffigures
\listoftables

\setcounter{secnumdepth}{0}
\setcounter{tocdepth}{0}
\chapter{Prefacio}

Este documento, junto con el software que le acompa�a, es una herramienta
creada para el aprendizaje de Sistemas Operativos, su dise�o e implementaci�n.
Su nombre es EAFITOS cuyo significado es ``EAFIT Operating System'', EAFIT
es la universidad donde naci�, situada en Medell�n (Colombia).

Ya existen otros Sistemas Operativos creados con los mismos fines.
Los cuales pueden ser divididos en dos grupos seg�n funcionen directamente
sobre una m�quina real (ejemplo, Minix) o sobre una m�quina imaginaria que
se simula sobre otro SO (ejemplo, Nachos).

Al desarrollar Eafitos se decidi� hacerlo sobre una m�quina imaginaria ya
que eso reduce mucho los costes de implementaci�n y simplifica el producto
final, el cual ser� m�s f�cil de comprender y modificar por parte de los
estudiantes.


\section{Historia}

\subsection{La primera versi�n}
\emph{Los textos de esta secci�n son fragmentos extra�dos de la documentaci�n
de la primera versi�n de Eafitos.}

\subsubsection{Origen}
El Sistema Operativo EAFITOS fue dise�ado e implementado con el fin de
permitir que los estudiantes del curso de pregrado ST075, de la Universidad
de EAFIT (Medell�n, Colombia) tuvieran una herramienta (simple pero completa)
para comprender, visualizar, evaluar y experimentar los conceptos abstractos
que se ense�an en un curso b�sico de S.O.
La primera versi�n de este producto fue desarrollada por aproximadamente 10
grupos de 3 estudiantes en promedio durante el primer semestre de 1995 y
ocupa menos de un disquete de 1.44~Megas incluyendo programas fuentes y
ejecutables.
Posteriormente, en los cursos siguientes se han ido mejorando algunas de sus
funciones b�sicas como manejo de ventanas simples, procesos concurrentes,
sem�foros, etc...
Pero quiz� lo m�s importante es que se ha ido incrementando poco a poco la
calidad del mismo y en consecuencia el n�mero de ``bugs''.

\subsubsection{Objetivos}
Los principales objetivos que se tuvieron en cuenta en el dise�o del S.O.
EAFITOS fueron:
\begin{enumerate}
\item Ser tan simple que pueda ser entendido en pocas sesiones de clase con
	el fin de que el estudiante pueda evaluarlo, modificarlo y/o
	extenderlo como parte del trabajo normal del curso ST075.
\item Correr en el computador personal de cada estudiante sin obligarlo a
	hacer particiones de su disco ni ampliar el hardware.
\item Implementar conceptos b�sicos y fundamentales del S.O. que no tiene
	DOS-PC como son multiprogramaci�n, planificaci�n de procesos, hilos,
	mecanismos de sincronizaci�n, seguridad, adecuado nivel de
	protecci�n de archivos, protecci�n de memoria, etc..
\item Mostrar claramente la diferencia entre la especificaci�n de un S.O.
	(llamadas al sistema) y la implementaci�n de las mismas, procurando
	que la interfaz se mantenga cuando la implementaci�n cambie.
\item Facilitar a comprender los conceptos b�sicos de dise�o como complejidad,
	eficiencia, simplicidad, elegancia, robustez, etc.. y las relaciones
	entre ellos.
\item Ser f�cilmente transportable a otras plataformas de S.O.
\item Poder usarse como herramienta en otros cursos de la carrera de
	Ingenier�a de Sistemas como son Teor�a de Archivos, Bases de Datos,
	Arquitectura de Computadores, Telem�tica, Compiladores, etc..
\end{enumerate}


\subsection{La versi�n 2.0}
En el verano de 1998 se me concedi� una beca del programa de Cooperaci�n
Interuniversitaria, que otorga la AECI (Agencia Espa�ola para la Cooperaci�n
Internacional) a estudiantes espa�oles para realizar trabajos en universidades
Latino-Americanas.

El destino de la beca fue la universidad EAFIT de Medell�n (Colombia) y el
objetivo de la misma era continuar con el desarrollo del sistema operativo
educacional EAFITOS.
Mi tutor en Colombia fue el profesor Omar Garc�a Palencia.
Decidimos reescribir EAFITOS desde cero, para conseguir un software que
se adecuara mejor a sus objetivos educativos.

No fue posible terminar el trabajo durante los dos meses que dur� mi estancia
en Colombia, as� que lo termin� en Espa�a, aprovech�ndolo adem�s como proyecto
final de carrera (mi director de proyecto fue el profesor Rafael Mayo Gual).

La nueva versi�n de Eafitos se adecua mejor a sus objetivos de dise�o.
Por ejemplo, es m�s portable, funciona en DOS y en GNU/Linux y portarlo a
otro sistema no deber�a implicar la modificaci�n de una sola l�nea de c�digo
fuente; tambi�n es m�s f�cil de comprender y modificar, gracias en parte a
un dise�o m�s cuidado y a una documentaci�n de mayor calidad.

Pero adem�s de esto, un objetivo fundamental de la nueva versi�n de Eafitos
era acercarse un poco m�s a los sistemas operativos reales, ya que la primera
versi�n ten�a algunas diferencias graves e innecesarias.
Por ejemplo, la nueva versi�n elimina los registros de tipo cadena del
procesador e implementa las llamadas al sistema como tales (no como
instrucciones del procesador).

\subsection{Algunas preguntas y sus respuestas}
\subsubsection{�Por qu� orientado a objetos?}
El programa, de por s�, requiere un enfoque orientado a objetos (para la
abstracci�n de sistemas de ficheros, por ejemplo).
Esto se puede implementar mediante punteros con un lenguaje m�s tradicional,
como C, de hecho eso es lo que hacen muchos n�cleos de sistema operativo
reales, Linux por ejemplo.
Pero resulta mucho m�s natural hacerlo con un lenguaje orientado a objetos.

Adem�s, estos lenguajes suelen ofrecer otras caracter�sticas que hacen m�s
f�cil la vida de los estudiantes y desarrolladores.
As�, se simplifica el dise�o y se incrementa la legibilidad del c�digo.


\subsubsection{�Por qu� C++?}
La versi�n original de Eafitos estaba escrita en ANSI C, igual que la
mayor�a de sistemas operativos reales.
Pero C no nos vale porque no es orientado a objetos.

Probablemente otros lenguajes hubieran servido a nuestros prop�sitos, de
hecho, inicialmente se consider� la posibilidad de escribir la nueva versi�n
en Java.
Pero pensamos que era mejor utilizar un lenguaje compilado para que a ning�n
estudiante se le quedase ``peque�o'' su ordenador.

Existen otros lenguajes compilados que podr�an haber servido, podr�amos
haberlos evaluado con detenimiento pero eso hubiera significado mucho
tiempo.
As� que elegimos C++, porque sirve perfectamente a nuestros prop�sitos y
porque pasar de C a C++ resulta ser lo m�s natural.

\subsubsection{�Por qu� Curses?}
La versi�n original de Eafitos utilizaba la librer�a de DOS \emph{conio}
para implementar el editor de textos y el depurador.
Pero esta librer�a tan solo est� disponible, que yo sepa, en DOS.

Para mejorar la portabilidad del c�digo se decidi� utilizar Curses, una
l�breria que ha sido portada a un gran n�mero de sistemas distintos.
En concreto, se ha utilizado \emph{NCurses} en GNU/Linux y \emph{PDCurses}
en DOS.


\section{Acerca de los responsables del proyecto}

El profesor Omar Garc�a Palencia es quien empez� todo esto.
Es profesor de Ingenier�a de Sistemas en la Universidad EAFIT de Medell�n
(Colombia).
Puede ser localizado en:
\begin{quote}
\begin{tabular}{ll}
\textbf{Correo electr�nico} & : \emph{ogarcia@sigma.eafit.edu.co}\\
\textbf{P�ginas web} & : \emph{http://sigma.eafit.edu.co/\~{}ogarcia}
\end{tabular}
\end{quote}

El autor de este documento y de la versi�n 2.0 de Eafitos es Juan David
Ib��ez Palomar.
Soy Ingeniero en Inform�tica de la Universidad Jaume~I de Castell�n (Espa�a).
Puedes encontrarme en:
\begin{quote}
\begin{tabular}{ll}
\textbf{Correo electr�nico} & : \emph{al004151@alumail.uji.es}\\
\textbf{P�ginas web} & : \emph{http://www4.uji.es/\~{}al004151}
\end{tabular}
\end{quote}

\setcounter{secnumdepth}{2}
\setcounter{tocdepth}{2}


\mainmatter

\part{Usar y programar Eafitos}
\chapter{Manual de usuario}

\section{Instalaci�n de Eafitos}
Para poder ejecutar Eafitos tan solo necesitas el fichero ejecutable, el cual
debes haberlo obtenido junto con este documento.

Pero antes de ejecutar Eafitos debes tener en cuenta una cosa.
Eafitos necesita de unos ficheros (los discos duros virtuales) que se
alojaran en el directorio ``$\backslash$EAFITOS'' o ``\$HOME/EAFITOS''
seg�n utilices DOS o GNU/Linux.
Asegurate de tener dicho directorio creado antes de ejecutar Eafitos.

\subsubsection{Compilaci�n}
Para compilar Eafitos en DOS necesitar�s Borland C++ 4.5, es posible
que funcione con otras versiones (posteriores a la versi�n 3.1), pero no
puedo garantizarlo ya que no lo he probado.

Para compilarlo en GNU/Linux necesitar�s el compilador \emph{egcs} y la
librer�a \emph{NCURSES}.
Si dispones de esto, es posible que incluso puedas compilarlo en otras
versiones de UNIX.

La compilaci�n en otros entornos ser� posible siempre y cuando dispongas
de alguna l�breria de Curses y de un compilador de C++ que implemente,
al menos, la revisi�n n�mero 3.0 del borrador oficial de ANSI C++.

\section{Gesti�n de los discos}
La informaci�n en Eafitos se almacena en unos discos virtuales que en realidad
son ficheros del Sistema Operativo real sobre el que se ejecuta Eafitos.

Cuando ejecutas Eafitos aparece un men� con cuatro opciones.
La primera te permitir� arrancar el int�rprete de comandos para empezar a
trabajar, pero antes de hacer esto debes tener al menos un disco creado y
formateado.
Para eso tienes las opciones 2 y 3, primero crea los discos que quieras
(hasta un m�ximo de cinco) con la opci�n 2 y despu�s les das formato;
ahora ya puedes iniciar el int�rprete.
La �ltima opci�n, la cuarta, es, como su nombre indica, para terminar la
ejecuci�n de Eafitos.

\section{El int�rprete de comandos}
Ahora vamos a ver qu� es lo que se puede hacer con Eafitos.
Para ello estudiaremos su int�rprete de comandos.

Primero, has de saber que el int�rprete distingue entre may�sculas y
min�sculas (como UNIX y al contrario que DOS), por lo que `Salir' no es
lo mismo que `salir'.

Ahora veamos la lista de comandos que nos proporciona:
\begin{itemize}
\item \textbf{salir} Termina el int�rprete y regresa al men� inicial.
\item \textbf{disco {\em numero}} Cambia el disco actual al indicado y el
	directorio actual pasa a ser el directorio ra�z del nuevo disco.
\item \textbf{creadir {\em nombre}} Crea el directorio {\em nombre}.
\item \textbf{cd {\em directorio}} Cambia el directorio actual al indicado.
\item \textbf{poner {\em fichero}} Pasa el fichero indicado del SO anfitri�n
	a Eafitos.
\item \textbf{obtener {\em fichero}} Pasa el fichero indicado de Eafitos al SO
	anfitri�n.
\item \textbf{borrar {\em fichero}} Borra el fichero indicado (tambi�n sirve
	para direcotorios).
\item \textbf{dir} Lista el contenido del directorio actual.
\item \textbf{ver {\em fichero}} Muestra el contenido del fichero indicado
	(como \emph{type} de DOS o \emph{cat} de UNIX).
\item \textbf{compilar {\em fichero}} Compila el fichero indicado (aqu�
	entra en juego el compilador), el ejecutable que se genera recibe
	el nombre {\em fichero.exe}.
\end{itemize}

Cuando escribes cualquier otra cadena, si corresponde a un fichero ejecutable
lo ejecuta y si no da un mensaje de error.
Los programas de usuario no admiten par�metros en la l�nea de comandos.
Cualquier par�metro de m�s en la l�nea de comandos se ignora.

\chapter{Manual del programador}
En este cap�tulo veremos todo lo que se necesita saber para escribir
programas para el sistema operativo Eafitos, y algo m�s.

La CPU de la m�quina virtual sobre la que se ejecuta Eafitos entiende un
c�digo m�quina muy sencillo, con un conjunto de instrucciones muy reducido
que se describir� m�s adelante.
Pero el programador de Eafitos no tiene que tratar directamente con el
c�digo m�quina, ya que se ha desarrollado un lenguaje ensamblador y un
compilador que hacen m�s f�cil la vida del programador.

En las Secciones~\ref{Sec: Estruct. de un prog.}~y~\ref{Sec: Arquitectura}
se describe dicho lenguaje y se explica la arquitectura del procesador;
en la Secci�n~\ref{Sec: Llamadas al sistema} se describen los servicios que
nos proporciona el n�cleo de Eafitos; finalmente, en la
Secci�n~\ref{Sec: Ejemplo} se estudia un peque�o programa que servir� de
ejemplo.

\section{Estructura de un programa}
\label{Sec: Estruct. de un prog.}

Un programa de Eafitos consta de dos partes.
Primero una declaraci�n de variables, que es opcional, y a continuaci�n
el c�digo.

En la Figura~\ref{Fig: dos ejemplos} vemos dos peque�os ejemplos que nos
ayudar�n a comprender mejor dicha estructura.

\begin{figure}
\begin{verbatim}
; Ejemplo n�mero 1                 ; Ejemplo n�mero 2
DATOS                              CODIGO
        numero  #3                         cargar_i @0, #2
        cadena  "hola"                     cargar_i @1, #3
CODIGO                                     sumar @0, @1, @3
        cargar_i @0, #2
etiq    cargar_i @1, numero
        cargar32 @1, @1
        sumar @0, @1, @3
\end{verbatim}
\caption{Ejemplos de programas de Eafitos}
\label{Fig: dos ejemplos}
\end{figure}

Lo primero que se observa en ambos ejemplos es una l�nea empezada por un
punto y coma.
Se trata de un comentario. 
Los comentarios empiezan por punto y coma y acaban en el final de la l�nea.
Se pueden poner en cualquier lugar del programa.

Lo que diferencia a ambos ejemplos es que el segundo no tiene una zona de
declaraci�n de variables.
La zona de declaraci�n de variables, si existe, debe estar siempre al inicio
del programa y empezar por la palabra DATOS (o DaTos, el compilador no es
sensible a may�sculas/min�sculas).
En ella existen una lista de variables, primero el nombre de la variable y
a continuaci�n su valor inicial.
Existen dos tipos de variable:
\begin{itemize}
\item De tipo num�rico.
	Ocupan 32 bits.
	El valor inicial que toma la variable debe ir precedido por una
	almohadilla (\#).
\item De tipo cadena. Ocupan tanto como la longitud de la cadena m�s uno
	(por el car�cter de final de cadena).
	La cadena aparece entre comillas.
\end{itemize}

Las variables siempre deben inicializarse, aunque despu�s no se vaya a
utilizar el valor inicial que contienen.
M�s adelante, en el c�digo, referenciaremos las variables por su nombre,
esto no nos dar� el valor que contienen sino la direcci�n de memoria que
representan.

A continuaci�n va el c�digo, el cual debe empezar por la palabra CODIGO.
El c�digo no es m�s que una lista de instrucciones, una en cada l�nea,
primero aparece el nombre de la instrucci�n y despu�s sus operandos
separados por comas.
Adem�s, opcionalmente, puede aparecer al principio una etiqueta que se podr�
utilizar como destino en los saltos.
En el primer ejemplo se observa una, aunque en este caso no se llega a
utilizar.

En cuanto a los operandos, podemos observar tres formas diferentes de
indicarlos:
\begin{itemize}
\item Registros.
	Primero se escribe un car�cter arroba (@) y despu�s el n�mero
	del registro
\item Inmediato.
	Primero aparece un car�cter almohadilla (\#) y despu�s el valor del
	dato.
\item Referencia a variable.
	Tan solo aparece el nombre de la variable.
\end{itemize}


\section{Arquitectura del procesador}
\label{Sec: Arquitectura}

En esta secci�n estudiaremos el juego de instrucciones con cierto detalle,
y veremos todos los aspectos del procesador que interesan al programador
de Eafitos.

\subsection{Registros del procesador}
\label{Sec: Registros}
Este es un procesador basado en registros generales, tiene un total de 16
numerados del 0 al 15, que el programador utilizar� expl�citamente.
Adem�s, cuenta con los t�picos registros de contador de programa y puntero
a pila, que se modifican impl�citamente con las instrucciones de salto y
las de gesti�n de la pila respectivamente.

Por otro lado, carece de registro de estado.
Los saltos condicionales se realizan en funci�n del valor de los registros
generales.

\subsection{Formato de las instrucciones}
La longitud de las instrucciones es variable.
El primer byte identifica de que instrucci�n se trata (por lo tanto, pueden
haber hasta 256 instrucciones).
A continuaci�n vienen los operandos, que pueden ser de dos tipos:

\begin{description}
\item [Registro] Referencia a alguno de los 16 registros del procesador.
	Ocupa un byte.
\item [Inmediato] Es un dato de 32 bits.
\end{description}


Existen seis tipos distintos de instrucciones seg�n su n�mero y tipo de
operandos.
Son:

\begin{itemize}
\item Sin operandos. Solo ocupan un byte.
\item Con un operando de tipo registro. Ocupan dos bytes.
\item Dos operandos de tipo registro. Tres bytes.
\item Tres operandos de tipo registro. Cuatro bytes.
\item Un operando de tipo inmediato. Cinco bytes.
\item Un operando de tipo inmediato y otro de tipo registro. Seis bytes.
\end{itemize}


\subsection{Modos de direccionamiento}
Existen cuatro modos de direccionamiento:
\begin{description}
\item [Directo a registro]
	El valor se encuentra almacenado en el registro.
	Un registro, un byte.
\item [Inmediato]
	El valor esta contenido en la propia instrucci�n.
	Un dato inmediato, cuatro bytes.
\item [Indirecto mediante registro]
	El registro especifica la direcci�n de memoria donde se encuentra
	el valor.
	Un registro, un byte.
\item [Directo a memoria]
	La propia instrucci�n especifica la direcci�n de memoria donde se
	encuentra el valor.
	Un dato inmediato, cuatro bytes.
\end{description}


\subsection{Descripci�n individual de todas las instrucciones}

\subsubsection{Aritm�tico/L�gicas}
Pueden ser instrucciones de dos o de tres operandos, seg�n el operador que
representan sea unario o binario respectivamente.
En cualquier caso, el modo de direccionamiento utilizado para todos los
operandos es el directo a registro.

Los primeros operandos (uno o dos seg�n sea una operaci�n unaria o binaria)
son los registros fuente sobre los que se realiza la operaci�n.
El �ltimo es el registro destino donde se almacena el resultado.

Existen seis instrucciones en esta categor�a:

\begin{itemize}
\item \textbf{sumar \emph{fuente1}, \emph{fuente2}, \emph{destino}}
	\begin{verse}
	\emph{Descripci�n}: suma los dos operandos fuente y almacena el
		resultado en el destino.

	\emph{Modos de direccionamiento de los operandos}: directo a registro.
	\end{verse}

\item \textbf{restar \emph{fuente1}, \emph{fuente2}, \emph{destino}}
	\begin{verse}
	\emph{Descripci�n}: Resta el segundo operando fuente del primero y
		almacena el resultado en el destino

	\emph{Modos de direccionamiento de los operandos}: directo a registro.
	\end{verse}

\item \textbf{and \emph{fuente1}, \emph{fuente2}, \emph{destino}}
	\begin{verse}
	\emph{Descripci�n}: aplica la operaci�n Y l�gica a los dos operandos
		fuente y almacena el resultado en el destino.

	\emph{Modos de direccionamiento de los operandos}: directo a registro.
	\end{verse}

\item \textbf{or \emph{fuente1}, \emph{fuente2}, \emph{destino}}
	\begin{verse}
	\emph{Descripci�n}: aplica la operaci�n O l�gica a los dos operandos
		fuente y almacena el resultado en el destino.

	\emph{Modos de direccionamiento de los operandos}: directo a registro.
	\end{verse}

\item \textbf{copiar \emph{fuente}, \emph{destino}}
	\begin{verse}
	\emph{Descripci�n}: copia del registro fuente al destino.

	\emph{Modos de direccionamiento de los operandos}: directo a registro.
	\end{verse}

\item \textbf{not \emph{fuente}, \emph{destino}}
	\begin{verse}
	\emph{Descripci�n}: almacena en destino el resultado de aplicar la
		negaci�n l�gica al registro fuente.

	\emph{Modos de direccionamiento de los operandos}: directo a registro.
	\end{verse}
\end{itemize}

\subsubsection{De acceso a memoria}
Tienen siempre dos operandos.
El primero es siempre directo a registro.
El segundo puede ser cualquiera de los cuatro modos de direccionamiento,
excepto el directo a registro.

\begin{itemize}
\item \textbf{cargar32 \emph{destino}, \emph{fuente}}
	\begin{verse}
	\emph{Descripci�n}: lee de memoria (fuente) un dato de 32 bits y lo
		almacena en un registro (destino).

	\emph{Modo de direccionamiento del destino}: directo a registro.

	\emph{Modo de direccionamiento del fuente}: indirecto mediante
		registro.
	\end{verse}

\item \textbf{guardar32 \emph{fuente}, \emph{destino}}
	\begin{verse}
	\emph{Descripci�n}: escribe en memoria (destino) un dato de 32 bits
		que est� almacenado en un registro(fuente).

	\emph{Modo de direccionamiento del fuente}: directo a registro.

	\emph{Modo de direccionamiento del destino}: indirecto mediante
		registro.
	\end{verse}

\item \textbf{cargar8 \emph{destino}, \emph{fuente}}
	\begin{verse}
	\emph{Descripci�n}: lee de memoria (fuente) un dato de 8 bits y lo
		almacena en un registro (destino).

	\emph{Modo de direccionamiento del destino}: directo a registro.

	\emph{Modo de direccionamiento del fuente}: indirecto mediante
		registro.
	\end{verse}

\item \textbf{guardar8 \emph{fuente}, \emph{destino}}
	\begin{verse}
	\emph{Descripci�n}: escribe en memoria (destino) un dato de 8 bits
		que est� almacenado en un registro(fuente).

	\emph{Modo de direccionamiento del fuente}: directo a registro.

	\emph{Modo de direccionamiento del destino}: indirecto mediante
		registro.
	\end{verse}

\item \textbf{cargar\_i \emph{destino}, \emph{fuente}}
	\begin{verse}
	\emph{Descripci�n}: lee de memoria (fuente) un dato que almacena en
		un registro(destino).

	\emph{Modo de direccionamiento del destino}: directo a registro.

	\emph{Modo de direccionamiento del fuente}: inmediato.
	\end{verse}

\item \textbf{guardar\_i \emph{fuente}, \emph{destino}}
	\begin{verse}
	\emph{Descripci�n}: escribe en memoria (destino) un dato que esta
		almacenado en un registro(fuente).

	\emph{Modo de direccionamiento del fuente}: directo a registro.

	\emph{Modo de direccionamiento del destino}: directo a memoria.
	\end{verse}
\end{itemize}


\subsubsection{De gesti�n de la pila}
En esta categor�a encontramos dos instrucciones que nos permiten almacenar
y recuperar datos en la pila del sistema.

La pila se utiliza para el paso de par�metros cuando realizamos llamadas al
sistema.
Tambi�n se podr�a usar para implementar subrutinas.

Tan solo tienen un operando de tipo registro, el otro est� impl�cito en el
registro especial {\em sp} (puntero a pila).
Por lo tanto solo ocupan dos bytes.

La pila crece hacia arriba, esto es, cuando almacenamos un dato se incrementa
el puntero a la pila y cuando lo extraemos se decrementa.

\begin{itemize}
\item \textbf{apilar \emph{fuente}}
	\begin{verse}
	\emph{Descripci�n}: almacena en la pila el operando {\em fuente}.

	\emph{Modo de direccionamiento del operando}: directo a registro.
	\end{verse}

\item \textbf{desapilar \emph{destino}}
	\begin{verse}
	\emph{Descripci�n}: extrae un dato de la pila y lo almacena en
		{\em destino}.

	\emph{Modo de direccionamiento del operando}: directo a registro.
	\end{verse}
\end{itemize}


\subsubsection{De control de flujo}
Aqu� est�n los saltos, que pueden tener uno o dos operandos seg�n sean 
incondicionales o condicionales.
Todo salto tiene siempre como �ltimo operando el destino del salto, que
lo normal es que se trate de una etiqueta.
Los saltos condicionales tienen, adem�s, un operando de tipo registro,
en funci�n de cuyo valor se realizar� o no el salto.
En total son cuatro:

\begin{itemize}
\item \textbf{saltar \emph{destino}}
	\begin{verse}
	\emph{Descripci�n}: salta incondicionalmente a la direcci�n
		{\em destino}.

	\emph{Modo de direccionamiento del destino}: inmediato.
	\end{verse}

\item \textbf{saltar0 \emph{fuente}, \emph{destino}}
	\begin{verse}
	\emph{Descripci�n}: salta a la direcci�n {\em destino} solo si el
		operando {\em fuente} es cero.

	\emph{Modo de direccionamiento del fuente}: directo a registro.

	\emph{Modo de direccionamiento del destino}: inmediato.
	\end{verse}

\item \textbf{saltarP \emph{fuente}, \emph{destino}}
	\begin{verse}
	\emph{Descripci�n}: salta a la direcci�n {\em destino} solo si el
		operando {\em fuente} es mayor que cero.

	\emph{Modo de direccionamiento del fuente}: directo a registro.

	\emph{Modo de direccionamiento del destino}: inmediato.
	\end{verse}

\item \textbf{saltarN \emph{fuente}, \emph{destino}}
	\begin{verse}
	\emph{Descripci�n}: salta a la direcci�n {\em destino} solo si el
		operando {\em fuente} es menor que cero.

	\emph{Modo de direccionamiento del fuente}: directo a registro.

	\emph{Modo de direccionamiento del destino}: inmediato.
	\end{verse}
\end{itemize}


\subsubsection{Especiales}
Aqu� hay dos, y ninguna de ellas tiene operandos:

\begin{itemize}
\item \textbf{nop}
	\begin{verse}
	\emph{Descripci�n}: instrucci�n de no operaci�n, no hace nada.
	\end{verse}

\item \textbf{ser\_sis}
	\begin{verse}
	\emph{Descripci�n}: llamada al sistema, pasa el control al n�cleo.
		Los par�metros se pasan a trav�s de la pila, ver la
		Secci�n~\ref{Sec: Llamadas al sistema} para m�s detalles.
	\end{verse}
\end{itemize}


\section{Llamadas al sistema}
\label{Sec: Llamadas al sistema}

En esta secci�n vamos a ver como realizar una llamada al sistema y qu�
servicios proporciona el n�cleo de Eafitos al programador.

\subsection{Realizando una llamada}
El n�cleo proporciona un conjunto de servicios que el programador puede
utilizar mediante la instrucci�n {\em ser\_sis}.
Pero antes de ejecutar dicha instrucci�n el programador debe almacenar en
la pila una serie de par�metros, adecuadamente ordenados.
El n�mero y significado de dichos par�metros depender�n del servicio que
se solicite, en cualquier caso siempre existe al menos uno.
El �ltimo par�metro que se introduce en la pila identifica el servicio
concreto que requerimos del n�cleo.
Cada par�metro se almacena directamente en la pila si cabe (si es de 32 o
menos bits), pero existen casos en los que se almacena un puntero a la zona
de memoria donde se encuentra el par�metro, esto sucede cuando ocupa m�s
de cuatro bytes, por ejemplo en el caso de cadenas de caracteres.
Adem�s, el n�cleo siempre devuelve un resultado en el registro 0.

\subsection{Descripci�n individual de los servicios del sistema}

\subsubsection{Gesti�n de hilos}
\begin{itemize}
\item \textbf{Crear hilo}
	\begin{verse}
	\emph{C�digo}: 1

	\emph{Par�metro}: Direcci�n en la que empezar� a ejecutarse el nuevo
		hilo.

	\emph{Resultado}: El identificador del nuevo hilo o -1 si la
		operaci�n ha fracasado.

	\emph{Descripci�n}: Crea un nuevo hilo del proceso en ejecuci�n.
	\end{verse}

\item \textbf{Terminar hilo}
	\begin{verse}
	\emph{C�digo}: 2

	\emph{Par�metro}: identificador del proceso que se quiere eliminar.

	\emph{Resultado}: �xito (0) o fracaso (-1) de la operaci�n.

	\emph{Descripci�n}: termina el hilo que se especifica.
	\end{verse}

\item \textbf{Terminar}
	\begin{verse}
	\emph{C�digo}: 3

	\emph{Resultado}: �xito (0) o fracaso (-1) de la operaci�n.

	\emph{Descripci�n}: termina el hilo actual (no tiene par�metros).
	\end{verse}
\end{itemize}

\subsubsection{Sistema de ficheros}
\begin{itemize}
\item \textbf{Cambiar unidad}
	\begin{verse}
	\emph{C�digo}: 10

	\emph{Par�metro}: Identificador de la nueva unidad.

	\emph{Resultado}: �xito (0) o fracaso (-1) de la operaci�n.

	\emph{Descripci�n}: Cambia la unidad de disco actual en la cual se
		est� ejecutando el hilo.
	\end{verse}

\item \textbf{Cambiar directorio}
	\begin{verse}
	\emph{C�digo}: 11

	\emph{Par�metro}: Ruta al nuevo directorio (puntero).

	\emph{Resultado}: �xito (0) o fracaso (-1) de la operaci�n.

	\emph{Descripci�n}: Cambia el directorio de actual en el que se
		est� ejecutando el hilo. El nuevo directorio se especifica
		con una cadena, que puede ser una ruta absoluta, si empieza
		por `/', o relativa al directorio actual si no.
	\end{verse}

\item \textbf{Crear fichero}
	\begin{verse}
	\emph{C�digo}: 12

	\emph{Par�metro 1}: Nombre del fichero (puntero).

	\emph{Par�metro 2}: Tipo del fichero, normal (0) o directorio (1).

	\emph{Resultado}: Identificador del nuevo fichero o -1 si la
		operaci�n ha fracasado.

	\emph{Descripci�n}: Crea el fichero especificado en la unidad de disco
		actual. Si el fichero ya existe da un error.
	\end{verse}

\item \textbf{Abrir fichero}
	\begin{verse}
	\emph{C�digo}: 13

	\emph{Par�metro}: Nombre del fichero (puntero).

	\emph{Resultado}: Identificador del fichero o -1 si la operaci�n
		ha fracasado.

	\emph{Descripci�n}: Abre el fichero especificado.
	\end{verse}

\item \textbf{Cerrar fichero}
	\begin{verse}
	\emph{C�digo}: 14

	\emph{Par�metro}: Identificador del fichero.

	\emph{Resultado}: �xito (0) o fracaso (-1) de la operaci�n.

	\emph{Descripci�n}: Cierra el fichero especificado.
	\end{verse}

\item \textbf{Borrar fichero}
	\begin{verse}
	\emph{C�digo}: 15

	\emph{Par�metro}: Nombre del fichero (puntero).

	\emph{Resultado}: �xito (0) o fracaso (-1) de la operaci�n.

	\emph{Descripci�n}: Borra el fichero especificado.
	\end{verse}

\item \textbf{Leer fichero}
	\begin{verse}
	\emph{C�digo}: 16

	\emph{Par�metro 1}: Identificador del fichero.

	\emph{Par�metro 2}: Direcci�n de memoria donde se almacenar� la
		informaci�n.

	\emph{Par�metro 3}: N�mero de bytes a leer.

	\emph{Resultado}: N�mero de bytes le�dos o -1 si la operaci�n ha
		fracasado.

	\emph{Descripci�n}: Lee un n�mero de bytes dado del fichero
		especificado.
	\end{verse}

\item \textbf{Escribir fichero}
	\begin{verse}
	\emph{C�digo}: 17

	\emph{Par�metro 1}: Identificador del fichero.

	\emph{Par�metro 2}: Direcci�n de memoria origen de la informaci�n.

	\emph{Par�metro 3}: N�mero de bytes a escribir.

	\emph{Resultado}: N�mero de bytes escritos o -1 si la operaci�n ha
		fracasado.

	\emph{Descripci�n}: Lee un n�mero de bytes dado del fichero
		especificado.
	\end{verse}

\item \textbf{Saltar fichero}
	\begin{verse}
	\emph{C�digo}: 18

	\emph{Par�metro 1}: Identificador del fichero.

	\emph{Par�metro 2}: Desplazamiento.

	\emph{Par�metro 3}: Posici�n base.

	\emph{Resultado}: Nueva posici�n dentro del fichero o -1 si la
		operaci�n ha fracasado.

	\emph{Descripci�n}: Modifica la posici�n actual dentro del fichero
		especificado, la nueva posici�n se obtiene sumando el
		desplazamiento a la posici�n base, que puede ser el
		principio del fichero (0) o la posici�n actual (1).
	\end{verse}
\end{itemize}

\subsubsection{Ejecuci�n de programas}
\begin{itemize}
\item \textbf{Ejecutar}
	\begin{verse}
	\emph{C�digo}: 20

	\emph{Par�metro}: Nombre del fichero (puntero).

	\emph{Resultado}: Identificador del nuevo hilo o -1 si la operaci�n ha
		fracasado.

	\emph{Descripci�n}: Ejecuta el fichero especificado.
	\end{verse}
\end{itemize}

\subsubsection{Entrada/Salida}
\begin{itemize}
\item \textbf{Obtener car�cter}
	\begin{verse}
	\emph{C�digo}: 30

	\emph{Resultado}: Devuelve el car�cter le�do.

	\emph{Descripci�n}: Lee un car�cter de la entrada est�ndar.
	\end{verse}

\item \textbf{Imprimir car�cter}
	\begin{verse}
	\emph{C�digo}: 31

	\emph{Par�metro}: Car�cter que se quiere imprimir.

	\emph{Resultado}: Devuelve siempre 0.

	\emph{Descripci�n}: Imprime un car�cter en la salida est�ndar.
	\end{verse}
\end{itemize}


\section{Ejemplo}
\label{Sec: Ejemplo}
En la Figura~\ref{Fig: hola mundo} tenemos el t�pico programa de ejemplo que
escribe el mensaje ``�Hola mundo!'' en la pantalla.

\begin{figure}
	\verbatimtabinput{../../ejemplos/hola}
	\caption{Programa de ejemplo, escribe tres veces un mensaje.}
	\label{Fig: hola mundo}
\end{figure}

En primer lugar est� la declaraci�n de variables, en este caso solo hay una.
Su nombre es ({\em mensaje}), es de tipo cadena y representa el mensaje que
vamos a imprimir (observar que est� acabado en \verb|\n|, s�mbolo que el
compilador traduce a una nueva l�nea, el compilador tambi�n entiende el
s�mbolo \verb|\t| que representa un tabulador, como en C).

Despu�s se encuentra el c�digo, que se divide en dos partes, un bloque
de inicializaci�n y un bucle.

Primero cargamos el registro n�mero 1 con el n�mero del servicio del sistema
que vamos a utilizar, el que imprime un car�cter por las salida est�ndar
Despu�s cargamos el registro 2 con la direcci�n de memoria donde se almacena
la cadena.
Finalmente cargamos en el registro 3 una cantidad fija que sumaremos al
registro 2 para recorrer la cadena.

La etiqueta {\em bucle} identifica la direcci�n de memoria donde empieza el
bucle.
Empieza cargando en el registro 4 el car�cter que vamos a imprimir y
comprobamos que sea distinto de cero (el cero es el valor que representa el
final de una cadena).
Despu�s se realiza la llamada al sistema, primero apilamos los dos par�metros
necesarios (el car�cter que queremos imprimir, registro 4, y el n�mero del
servicio del sistema, registro 1) y despu�s ejecutamos la llamada.
Finalmente se incrementa el registro 2 para que apunte al siguiente car�cter
de la cadena y se salta al principio del bucle.
El bucle terminar� cuando se encuentre el car�cter de final de cadena
(valor 0), lo cual provoca un salto al final del programa
(etiqueta \verb fin ).


\section{Programaci�n avanzada}

\subsection{Manejo de Ficheros}
Todos los procesos tienen asociados una tabla de ficheros.
Cuando se abre o crea un fichero se utiliza una entrada de la tabla.
Las llamadas abrir y crear devuelven un identificador (un �ndice dentro de
la tabla) con el que despu�s se podr� utilizar para trabajar con el fichero.

Las llamadas al sistema relacionadas con el sistema de ficheros pueden ser
utilizadas con cualquier identificador de fichero.
Pero adem�s, las dos primeras entradas (identificadores 0 y 1) se usan de
forma impl�cita en las llamadas al sistema de entrada/salida; en concreto,
el fichero 0 es la entrada est�ndar, se utiliza en la llamada
\textbf{Obtener Car�cter}, y el fichero 1 es la salida est�ndar, se utiliza
en la llamada \textbf{Imprimir Car�cter}.
Por defecto la entrada est�ndar es el teclado y la salida est�ndar es la
pantalla.

\subsubsection{Redireccionar la entrada/salida est�ndar}
Cuando se abre o se crea un fichero el identificador devuelto se corresponde
siempre con el de la primera entrada libre de la tabla.
As�, para redireccionar la entrada est�ndar lo que tenemos que hacer es
cerrar el fichero n�mero 0 e inmediatamente despu�s abrir (o crear) el
fichero que queremos sea la nueva entrada est�ndar.

El fichero \emph{ejemplos/es} es un programa de ejemplo que ilustra como
redireccionar la salida est�ndar.

\subsubsection{Acceso a dispositivos de tipo car�cter}
Los dispositivos de tipo car�cter se pueden acceder como si se tratara de
ficheros normales.
Cada dispositivo de tipo car�cter tiene un nombre especial que podemos
utilizar para abrirlo.
El nombre del teclado es ``--teclado--'' y el nombre de la pantalla es
``--pantalla--''.


\part{Eafitos por dentro}
\chapter{Introducci�n}

\section{Dise�o global}
La diferencia m�s importante de Eafitos con los sistemas operativos reales es
el hecho de que Eafitos se ejecuta sobre una m�quina virtual en lugar de
directamente sobre el hardware.
Esto simplifica enormemente las cosas y tiene otras consecuencias como ahora
veremos.

Los n�cleos de los sistemas operativos reales suelen estar escritos en C y en
lenguaje ensamblador.
Entonces, hay que compilar el c�digo fuente para generar el binario que se
ejecutar� en la m�quina.
De este modo, el n�cleo es un programa m�s (aunque con unas caracter�sticas
particulares) que, igual que los programas de usuario, reside en la memoria
del sistema y cuyas instrucciones de c�digo m�quina ejecuta el procesador.

En Eafitos esto no es as�.
Eafitos est� completamente escrito en C++ y se ejecuta sobre un sistema
operativo anfitri�n (DOS o UNIX).
As�, mientras que los programas de usuario residen en la memoria virtual de
la m�quina virtual y son ejecutados por su procesador, el n�cleo y el resto
de herramientas (compilador, int�rprete de comandos,...) se ejecutan
directamente sobre el sistema operativo anfitri�n.
Hacerlo como en la realidad, es decir, escribir el n�cleo (y el resto de
herramientas) en un lenguaje dado y compilarlo para generar c�digo m�quina
que el procesador virtual pudiese entender, hubiese tenido un coste tremendo
ya que hubiese implicado crear un compilador much�simo m�s complejo o escribir
el c�digo en ensamblador.

La Figura~\ref{Fig: Eafitos} muestra un esquema en el que nos apoyaremos para
acabar de entender el dise�o de Eafitos.
\begin{figure}
	\includegraphics[width=\textwidth]{figuras/eafitos.eps}
	\caption{Dise�o global de Eafitos.}
	\label{Fig: Eafitos}
\end{figure}

En el esquema, los programas de usuario est�n dibujados en el interior de la
m�quina virtual para subrayar el hecho de que residen en la memoria virtual.
El n�cleo de Eafitos es el encargado de gestionar la m�quina virtual y de
proporcionar a los programas de usuario una serie de facilidades.
La flecha que va de la m�quina virtual al n�cleo hace referencia a las
llamadas al sistema y a la interrupci�n de reloj (ver
Cap�tulo~\ref{Cap: Nucleo}).

Tres de las cuatro partes de Eafitos utilizan recursos directamente del
sistema operativo anfitri�n.
El entorno y el int�rprete de comandos lo hacen para interactuar con el
usuario, el compilador para presentar mensajes de error o de �xito y la
m�quina virtual para implementar sus dispositivos.

El usuario interact�a directamente tan solo con el entorno (men�s iniciales
para la gesti�n de los discos) y el int�rprete de comandos.
Es pues el int�rprete de comandos el que se ejecuta la mayor parte del
tiempo, el control solo pasa a la m�quina virtual cuando se ejecuta alg�n
programa.

\section{Estructura del c�digo fuente}
Dentro del directorio \emph{programa} tenemos el c�digo fuente que se
distribuye en los siguientes directorios:
\begin{itemize}
\item \emph{include} Todos los ficheros de cabecera residen en este directorio.
\item \emph{mv} Aqu� est�n los ficheros de implementaci�n de la m�quina
	virtual.
\item \emph{nucleo} Aqu� est�n los ficheros de implementaci�n del n�cleo.
	Este directorio tiene, adem�s, los siguientes subdirectorios:
	\begin{itemize}
	\item \emph{es} Aqu� encontrar�s los ficheros que implementan la
		entrada/salida.
	\item \emph{sf} Aqu� residen los ficheros que implementan el sistema
		de ficheros.
	\item \emph{ejec} Aqu� encontrar�s los ficheros que implementan los
		formatos de ficheros ejecutables.
	\end{itemize}
\item \emph{entorno} En este directorio est�n los ficheros que implementan
	el int�rprete de comandos, el compilador y el sistema de men�s para
	la gesti�n de los discos.
\end{itemize}

El fichero que contiene la funci�n \id{main} es \emph{programa/eafitos2.cpp}.


\section{Cuestiones de implementaci�n}

\subsection{Implementaci�n de las clases principales}
La Figura~\ref{Fig: Eafitos} muestra a Eafitos dividido en cuatro partes,
aunque podemos considerar que solo hay tres si unimos el compilador al
entorno y al int�rprete de comandos, ya que as� es como est� implementado.

Disponemos de tres clases principales: \id{MaqVirtual,} \id{Nucleo} y
\id{Entorno} cuyos miembros (atributos y m�todos) son est�ticos y p�blicos.
Es aconsejable que mires los ficheros de cabecera donde est�n definidos estas
tres clases: \emph{mv.h}, \emph{nucleo.h} y \emph{entorno.h}.

Estas clases, fundamentalmente hacen la funci�n de contenedoras de otros
objetos, los cuales pueden ser referenciados con, por ejemplo,
\id{Nucleo::xxx.yyy().}
De hecho, el c�digo contiene muchas de estas referencias.
Es algo as� como si fueran variables globales, esto simplifica mucho el
c�digo.

Adem�s, estas clases principales contienen, entre otras cosas, c�digo de
inicializaci�n y terminaci�n que se estudiar� con m�s detalle en los pr�ximos
cap�tulos.

\subsubsection{La funci�n \id{main}}
Mira tambi�n la funci�n \id{main} en el fichero \emph{eafitos2.cpp}.
Todo lo que hace es iniciar la m�quina virtual y en n�cleo, ceder el control
al entorno y terminar el n�cleo y la m�quina virtual.

\subsection{Gesti�n de errores}
En principio la gesti�n de errores estaba implementada como se suele hacer en
C, haciendo que los m�todos devolvieran un c�digo de error (normalmente -1)
cuando se detectase un error, y evaluando el c�digo devuelto por los m�todos.

Pero m�s adelante decid� utilizar los mecanismos de gesti�n de excepciones de
C++ (\id{throw,} \id{try} y \id{catch}) porque facilitan much�simo el trabajo.
En general la sentencia \id{throw} siempre devuelve una cadena, y los bloques
\id{try}-\id{catch} se limitan o a ignorar el error o a imprimir un mensaje
de error y abortar la operaci�n.
Esto es cierto excepto en un caso particular en la clase \id{Nucleo,} donde
\marginnote{Ver include/nucleo.h}
est�n definidas dos clases (\id{NoHayHilos} y \id{NoHayHilosListos}) que
se utilizan para poder distinguir la causa de la excepci�n y poder dar as�
un tratamiento espec�fico.

En cualquier caso, lo que te quiero decir es que si vas a modificar Eafitos
debes utilizar tambi�n los mecanismos de gesti�n de excepciones de C++, vale
la pena.

\chapter{La M�quina Virtual}

La M�quina Virtual simula un ordenador real.
\marginnote{Ver include/mv.h}
Consta de varios componentes: la memoria, la unidad de gesti�n de memoria,
el procesador y los dispositivos (actualmente un teclado, una pantalla y
uno o m�s discos duros).
En la Figura~\ref{Fig: Maquina Virtual} se puede observar un dibujo que
refleja la estructura de la m�quina virtual.
C�mo se puede ver, no hay relaci�n entre los dispositivos y el procesador;
en los sistemas reales existen diversos mecanismos (interrupciones, acceso
directo a memoria, puertos de entrada/salida,...) que en Eafitos se han
obviado para simplificar.

\begin{figure}
	\includegraphics[width=\textwidth]{figuras/maq_vir.eps}
	\caption{Estructura de la m�quina virtual.}
	\label{Fig: Maquina Virtual}
\end{figure}


\section{La memoria}
La memoria est� implementada como la clase \id{MemFis.}
\marginnote{Ver include/memfis.h}
Fundamentalmente consta de un vector de bytes, que representa a la memoria
en s�, y de los m�todos que permiten acceder a ella, a nivel de byte,
\id{leerByte} y \id{escribirByte}, o a nivel de palabra (cuatro bytes),
\id{leerPalabra} y \id{escribirPalabra.}


\section{Unidad de gesti�n de memoria}
La unidad de gesti�n de memoria, o MMU (Memory Management Unit), proporciona
\marginnote{Ver include/mmu.h}
facilidades al n�cleo para la gesti�n de memoria y es la responsable de
calcular las direcciones f�sicas de memoria a partir de direcciones l�gicas.

La gesti�n de memoria de la m�quina virtual sigue un modelo que combina
segmentaci�n y paginaci�n.
Existe una �nica tabla de segmentos (el vector \id{MMU::segmentos}) y cada
segmento define un conjunto de p�ginas (\id{Segmento::paginas}). 

\subsection{C�lculo de la direcci�n f�sica}
En la Figura~\ref{Fig: De logica a fisica} puedes ver un dibujo ilustrativo.
\marginnote{Ver MMU::logicaAFisica en include/mmu.cpp}
El \emph{selector de segmento} indica qu� segmento utilizar para el c�lculo.
Una vez tenemos el segmento se obtiene la p�gina l�gica mediante el
\emph{selector de p�gina}.
La p�gina l�gica nos da la p�gina f�sica.
Ahora calculamos la direcci�n f�sica con la siguiente ecuaci�n:
\begin{quote}\small
	(\emph{p�gina f�sica} * \emph{tama�o de p�gina}) +
	\emph{desplazamiento} = \emph{direcci�n f�sica}
\normalsize\end{quote} 
donde \emph{tama�o de p�gina} es una constante.

\begin{figure}
	\includegraphics[width=\textwidth]{figuras/log_fis.eps}
	\caption{Traducci�n de una direcci�n l�gica a una direcci�n f�sica}
	\label{Fig: De logica a fisica}
\end{figure}

Pero, �de donde sale la direcci�n l�gica?
El \emph{selector de segmento} es un registro especial de la CPU.
Existen dos de estos registros, se utiliza uno u otro dependiendo de la
instrucci�n que se ejecute; es por lo tanto una selecci�n impl�cita,
un mecanismo transparente para el programador.
En la Secci�n~\ref{Sec: Procesador} veremos esto con m�s detalle.

Por otro lado, el \emph{selector de p�gina} y el \emph{desplazamiento}
constituyen las direcciones de 32 bits que el programador maneja;
dichas direcciones est�n contenidas en registros de manipulables por el
programador: los registros generales, el contador de programa y el puntero
de pila.
El n�mero de bits que ocupan ambos campos, \emph{selector de p�gina} y
\emph{desplazamiento}, depende del \emph{tama�o de p�gina}, el cual est�
definido por la constante \id{TAM\_PAGINA.}
En principio el valor de dicha constante es 1024, por lo tanto el
\emph{selector de p�gina} ocupa 10 bits y el \emph{desplazamiento} ocupa 22.
Puedes modificar el \emph{tama�o de p�gina} con la �nica condici�n de que
sea potencia de 2.

\subsection{Reservar y liberar memoria}
Adem�s de la tabla de segmentos, la unidad de gesti�n de memoria contiene
un vector (\id{MMU::paginas}) que contiene informaci�n sobre cada p�gina
f�sica.
En concreto, la informaci�n almacenada nos indica si la p�gina est� libre u
ocupada.
Este vector, junto con el atributo \id{MMU::nPaginasLibres} (indica el
n�mero de p�ginas libres que quedan), le sirve a la MMU para ayudarse en
la gesti�n de la memoria.

\subsubsection{Solicitando memoria}
El m�todo \id{MMU::asignar} sirve para solicitar una determinada cantidad
de memoria.
La MMU reserva un segmento y las p�ginas f�sicas necesarias (si existe
suficiente memoria libre) y devuelve el selector del segmento que se ha
reservado.

Tambi�n se puede incrementar el \emph{uso} (\id{Segmento::uso}) de un
segmento que ya est� ocupado con el m�todo \id{MMU::reservar,} esto permite
la compartici�n de memoria.

\subsubsection{Liberando memoria}
El m�todo \id{MMU::liberar} reduce el uso del segmento indicado.
Si el segmento ya no se utiliza (su uso es cero) libera la memoria que tiene
asociada y el propio segmento.


\section{El procesador}
\label{Sec: Procesador}

La clase \id{CPU} representa al procesador.
\marginnote{Ver include/cpu.h}
El objetivo esencial de esta unidad es el de ejecutar instrucciones.

\subsection{Contexto}
\label{Sec: Contexto}
El procesador de la m�quina virtual tiene un conjunto de registros, algunos
de los cuales ya vimos en la Secci�n~\ref{Sec: Registros} desde el punto de
vista del programador.
Ahora vamos a verlos todos con m�s detalle.

Los registros est�n definidos en la clase \id{Contexto,} ya que forman el
contexto de ejecuci�n de un hilo.
\marginnote{Ver include/contexto.h}

Por un lado tenemos 16 registros generales (\id{Contexto::registros}) para
uso del programador.
Tambi�n hay un contador de programa (\id{Contexto::pc}) y un puntero de pila
(\id{Contexto::sp}).

Finalmente, tenemos dos registros especiales, \id{Contexto::codigo} y
\id{Contexto::pila;} estos registros son selectores de segmento.
El registro de \emph{pila} se utiliza en las instrucciones de acceso a la
pila y el de \emph{codigo} en el resto de instrucciones.
El programador no tiene forma alguna de modificar estos registros, ni
explicita ni impl�citamente.

\subsection{Acceso a memoria}
En la clase \id{CPU} hay varios m�todos definidos para acceder a memoria.
\marginnote{Ver mv/cpu.cpp}
Estos m�todos permiten leer y escribir bytes y palabras, son
\id{CPU::leerByte,} \id{CPU::escribirByte,} \id{CPU::leerPalabra} y
\id{CPU::escribirPalabra.}

\subsubsection{Lectura de instrucciones y operandos}
Por otro lado los m�todos \id{CPU::leerOpByte} y \id{CPU::leerOpPalabra} leen
de la memoria utilizando el {\em contador de programa} como direcci�n.
Estos m�todos se utilizan para leer las instrucciones y sus operandos
inmediatos.
Adem�s de leer de memoria actualizan autom�ticamente el {\em contador de
programa}.

\subsubsection{La pila}
A la pila se accede mediante los m�todos \id{CPU::apilar} y
\id{CPU::desapilar.}
Es en estos m�todos donde est� definido su comportamiento.
En concreto, la pila est� implementada para crecer hacia ``arriba'', es decir,
cuando se apila un dato el {\em puntero de pila} (\id{CPU::contexto.sp}) se
incrementa y cuando se desapila se decrementa.


\subsection{Ejecuci�n de instrucciones}
Los m�todos principales del procesador son \id{ejecutarPaso} y \id{ejecutar.}
El primero ejecuta una instrucci�n mientras que el segundo llama al primero
para ejecutar instrucciones hasta que no quede ninguna, es decir, hasta que
todos los hilos hayan terminado.
\footnote{La raz�n para distinguir entre dos m�todos es la de facilitar la
construcci�n de un depurador. As�, ser�a f�cil ejecutar instrucciones
``paso a paso''.}

Las excepciones que se puedan provocar como consecuencia de la ejecuci�n de
una instrucci�n (ej: fallo de p�gina) se capturan en \id{CPU::ejecutar} y
provocan la terminaci�n del hilo.

El m�todo \id{CPU::ejecutarPaso} es muy sencillo, aunque extenso.
Para empezar se ejecuta el m�todo \id{Nucleo::reloj,} el cual cede el control
al n�cleo.
Esto simula las interrupciones de reloj que se producen cada cierto tiempo
en las m�quinas reales, solo que en este caso se ``producen'' antes de
ejecutar cada instrucci�n.
�Y qu� es lo que hace el n�cleo en esas \emph{interrupciones de reloj}?
Eso se estudiar� cuando veamos el n�cleo, en los siguientes cap�tulos.

Una vez hecho esto, se pasa a ejecutar una instrucci�n.
Primero se adquiere el c�digo de instrucci�n; despu�s se decodifica, lo cual
consiste en saltar dentro del \id{switch} al lugar adecuado; finalmente se
adquieren los operandos y se ejecuta.
Mejor le hechas un vistazo al c�digo.


\section{Dispositivos}
\subsection{El teclado}
El teclado virtual sencillamente nos proporciona la posibilidad de leer un
car�cter del teclado real.
\marginnote{Ver include/teclado.h}

\subsection{La pantalla}
La pantalla virtual sencillamente nos proporciona la posibilidad de escribir
un car�cter en la pantalla real.
\marginnote{Ver include/pantalla.h}

\subsection{Los discos}
Este es el dispositivo m�s complicado.
Al contrario que el teclado y la pantalla, la m�quina virtual puede contener
varios discos, tantos como indica la constante \id{NUM\_DISCOS.}
\marginnote{Ver include/disco.h}

Un disco es, en la m�quina real, un fichero que reside en el directorio
``$\backslash$EAFITOS'' o ``\$HOME/EAFITOS'', dependiendo de si el Sistema
Operativo anfitri�n es DOS o GNU/Linux; el nombre del fichero es
DISCO\_\emph{n}, donde \emph{n} es el n�mero del disco (de 0 a
\id{NUM\_DISCOS}-1).

El disco se divide en bloques de igual tama�o.
El n�mero de bloques y su tama�o puede variar de un disco a otro, dichos
valores est�n especificados al principio del fichero real.
Tras estos dos valores vienen los bloques en s�, en la Figura~\ref{Fig: Disco}
tienes un esquema.
Una vez creado el disco no se puede cambiar ni su tama�o ni su n�mero de
bloques; si lo cambias desde el SO anfitri�n perder�s la informaci�n que
contenga esa unidad.
\begin{figure}
	\includegraphics[width=\textwidth]{figuras/disco.eps}
	\caption{Esquema de un disco.}
	\label{Fig: Disco}
\end{figure}

\subsubsection{Creaci�n de nuevos discos}
Para crear nuevos discos disponemos del m�todo \id{Disco::crearDisco,}
\marginnote{Ver mv/disco.cpp}
el cual recibe como par�metros el n�mero del disco, \emph{tama�o de bloque}
y el \emph{n�mero de bloques} que queremos tenga el nuevo disco.

\section{Arranque y parada}
La memoria, la unidad de gesti�n de memoria y el procesador no necesitan ser
inicializados.
Ni el teclado ni la pantalla, ya que estos dispositivos son ``est�ticos'',
es decir, siempre hay uno y solamente uno de cada.
Por el contrario, puede haber un n�mero variable de discos que hay que
detectar e inicializar en el arranque.

El m�todo \id{MaqVirtual::iniciar} es el responsable del arranque de la
m�quina y el m�todo \id{MaqVirtual::terminar} lo es de su parada.
\marginnote{Ver mv/mv.cpp}
El primero inicializa el vector \id{MaqVirtual::discos} utilizando el
constructor de la clase \id{Disco.}
El segundo (\id{terminar}) ``limpia'' dicho vector.


\section{Modificar la m�quina virtual}

\subsection{A�adir instrucciones al procesador}
Primero hay que a�adir la instrucci�n al tipo enumerado \id{Instrucciones}
y despu�s modificar el m�todo \id{CPU::ejecutarPaso} a�adiendo una nueva
entrada al \id{switch} que haga lo que corresponda.

Ten en cuenta que para poder utilizar las nuevas instrucciones en el lenguaje
ensamblador tambi�n tendr�s que modificar el compilador.
Adem�s, si no has colocado todas las nuevas instrucciones al final de
\id{Instrucciones,} tendr�s que eliminar los ejecutables y volver a
compilar los programas, ya que de lo contrario es probable que no funcionen.

\subsection{A�adir nuevos dispositivos}
Simplemente escribe un fichero de cabecera \emph{nombre.h} y uno de
implementaci�n \emph{nombre.cpp} donde se defina e implemente la clase
que representa al dispositivo; despu�s a�ade una instancia, o varias,
seg�n corresponda, de dicha clase a la maquina virtual (\id{MaqVirtual}).
Finalmente, a�ade el c�digo de inicializaci�n/creaci�n que sea necesario,
f�jate en el caso de los discos como ejemplo. 

\chapter{El n�cleo: Introducci�n}
\label{Cap: Nucleo}

El n�cleo se divide en cinco partes: el manejador de memoria, el manejador
de hilos y procesos, los dispositivos de tipo car�cter, los dispositivos de
tipo bloque y el sistema de ficheros virtual.
Dichos elementos est�n contenidos en la clase \id{Nucleo,} y los estudiaremos
\marginnote{Ver include/nucleo.h}
por separado en los cap�tulos siguientes.

\section{Iniciar y terminar}
La inicializaci�n y la terminaci�n del n�cleo consiste en la inicializaci�n
y terminaci�n de sus partes.
Los m�todos \id{iniciar} y \id{terminar} son los encargados
de hacer esta tarea.

\section{Llamadas al sistema}
Las llamadas al sistema proporcionan ciertos servicios que el programador
puede utilizar.
Adem�s de las cinco partes del n�cleo arriba se�aladas, existe un interfaz
de llamadas al sistema, el cual est� implementado en el m�todo
\id{Nucleo::llamada.}
\marginnote{Ver nucleo/nucleo.cpp}
La Figura~\ref{Fig: Llamadas} muestra un esquema de esto, como se puede
observar, \id{llamada} es una especie de multiplexor que, en funci�n del
valor almacenado en la cima de la pila, llama a otro m�todo de alg�n otro
subsistema del n�cleo.
\begin{figure}
	\includegraphics[width=\textwidth]{figuras/llamadas.eps}
	\caption[Esquema b�sico del n�cleo y las llamadas al sistema.]
		{Esquema b�sico del n�cleo y las llamadas al sistema.
		El dibujo est� muy simplificado, tan solo pretende
		ilustrar la relaci�n entre el procesador, el interfaz de
		llamadas y el resto del n�cleo. Las relaciones que muestran
		las flechas internas del n�cleo son incompletas.}
	\label{Fig: Llamadas}
\end{figure}

\subsection{A�adir nuevas llamadas}
A�adir una nueva llamada al sistema es f�cil.
Primero hay que definir una nueva constante para la nueva llamada, esto se
hace al principio del fichero ``include/nucleo.h''; despu�s se a�ade una
entrada dentro del bloque \id{switch} en el m�todo \id{Nucleo::llamada} que
se encuentra en el fichero ``nucleo/nucleo.cpp'', en dicho c�digo lo �nico
que se debe hacer es desapilar los par�metros y llamar al m�todo del n�cleo
que corresponda; finalmente, quedar�n por hacer las modificaciones necesarias
al resto del n�cleo para que la llamada haga lo que se quiera.


\section{Interrupciones de reloj}
\label{Sec: Reloj}
Los sistemas reales disponen de una interrupci�n de reloj que produce el
hardware siempre a intervalos de tiempo regulares.
Al producirse dicha interrupci�n deja de ejecutarse moment�neamente el
proceso que se estuviera ejecutando y se ejecuta una rutina de servicio.
Dicha rutina se dedica a realizar tareas varias que pueden desencadenar,
por ejemplo, el cambio del proceso en ejecuci�n.

En Eafitos esto est� simulado de la siguiente manera: antes de ejecutar
cada instrucci�n el procesador de la m�quina virtual
\footnote{Ver Secci�n~\ref{Sec: Procesador}.}
llama al m�todo \id{Nucleo::reloj,} esto es ``la generaci�n de la interrupci�n
de reloj''; y el m�todo \id{Nucleo::reloj} es ``la rutina de servicio de la
interrupci�n de reloj''.

El m�todo \id{Nucleo::reloj} lo que hace es llamar a otros tres m�todos:
\begin{quote}
\begin{itemize}
\item \id{DispositivosCaracter::gestionarColas}
\item \id{DispositivosBloque::gestionarColas}
\item \id{ManejadorProcesos::planificador}
\end{itemize}
\end{quote}

Lo qu� hacen estos m�todos lo veremos en los cap�tulos siguientes.

\chapter{El n�cleo: gesti�n de memoria}

La parte del n�cleo responsable de la gesti�n de memoria est� representada
por la clase \id{ManejadorMemoria.}
\marginnote{Ver include/memoria.h}
El manejador de memoria se apoya en la unidad de gesti�n de memoria de la
m�quina virtual para realizar su tarea, de hecho la mayor parte del trabajo
lo realiza la \emph{MMU}.


\section{Reservar y liberar memoria}
Esta es la tarea principal del manejador de memoria.
Para realizarla disponemos de tres m�todos, \id{asignar,} \id{reservar} y
\id{liberar.}
Estos m�todos lo �nico que hacen es llamar a sus equivalentes en la unidad
de gesti�n de memoria.
El primero, \id{asignar,} reserva la memoria solicitada y devuelve el
\emph{selector de segmento} reservado; el m�todo \id{reservar} incrementa el
uso del segmento especificado; finalmente, \id{liberar} reduce el uso del
segmento especificado y, si ya no se utiliza (su uso es cero), libera al
segmento y a la memoria asociada.

\section{Copiar desde y hacia espacio de usuario}
En los sistemas operativos reales suelen existir unas funciones que copian
datos de memoria de usuario a memoria del n�cleo y viceversa.
En Eafitos tambi�n, solo que la memoria del n�cleo no reside en la memoria
de la m�quina virtual, sino directamente en la memoria del sistema operativo
anfitri�n.

Estos m�todos son \id{aUsuario} y \id{deUsuario}, los cuales copian,
respectivamente, datos de memoria del n�cleo a memoria de usuario y viceversa.
\marginnote{Ver nucleo/memoria.cpp}
Para se ello utilizan los m�todos \id{CPU::escribirByte} y \id{CPU::leerByte}.

\section{Modificar el modelo de memoria}
Modificar el modelo de memoria implica no solo modificar el manejador de
memoria, sino sobre todo modificar la unidad de gesti�n de memoria en la
m�quina virtual, ya que la mayor complejidad reside en la \emph{MMU}.
De hecho, dependiendo de los cambios que se quieran realizar, es probable que
no sea necesario modificar el manejador de memoria en nada.
El modelo de memoria no est� pensado especialmente para que sea f�cilmente
modificable, pero algunas cambios sencillos no deber�an ser dif�ciles de
hacer, ya que los mecanismos b�sicos de paginaci�n y segmentaci�n est�n
implementados.

\chapter{El n�cleo: hilos y procesos}
\label{Cap: Hilos}
Eafitos distingue perfectamente entre hilos y procesos.
Los hilos son los elementos activos del sistema operativo mientras que los
procesos son un conjunto de recursos.
Un hilo tiene siempre asociado un �nico proceso, mientras que un proceso
puede tener asociados uno o m�s hilos.

Cada uno de estos elementos, hilos y procesos, est�n definidos como clases,
\id{Hilo} y \id{Proceso} respectivamente.
Existen tambi�n otras clases relacionadas directamente con la gesti�n de
hilos y de procesos, tales como \id{Cola} y \id{ManejadorProcesos.}

\section{Hilos}

\subsection{Estados}
Un hilo puede estar en uno de tres estados, tal como muestra la
Figura~\ref{Fig: Estados de un hilo}.
\marginnote{Ver include/hilo.h}

\begin{figure}
	\includegraphics[width=\textwidth]{figuras/hilo.eps}
	\caption{Estados de un hilo}
	\label{Fig: Estados de un hilo}
\end{figure}

Cuando se crea un hilo este se encuentra inicialmente en el estado
{\em Listo}, es decir se encuentra disponible para ser ejecutado.
Pueden haber varios hilos en estado {\em Listo}.

En un momento dado el hilo que se encuentra en {\em Ejecuci�n} en estos
momentos pasar� a estado {\em Listo} y uno de los hilos que se encuentra
en {\em Listo} pasar� a {\em Ejecuci�n}.
Solo puede haber un hilo en {\em Ejecuci�n}, que es aquel que est� siendo
ejecutado en estos momentos por la CPU.
La pol�tica seg�n la cual se realiza este paso de {\em Listo} a
{\em Ejecuci�n} y viceversa se implementa en el planificador, que
estudiaremos m�s adelante.

En un momento dado el hilo que se encuentra en {\em Ejecuci�n} puede realizar
una petici�n de entrada/salida, esto har� que dicho hilo pase a estado
{\em Suspendido}  y que otro hilo en {\em Listo} pase a {\em Ejecuci�n}.
Cuando la petici�n de entrada/salida se haya satisfecho el hilo que la
realiz� pasar� a {\em Listo}.

Existen dos formas de que un hilo termine: de forma natural, cuando acaba
su ejecuci�n (como muestra la flecha que sale de {\em Ejecuci�n} hacia
ninguna parte); y de forma ``violenta'' cuando alg�n otro hilo le ``mata'',
esta segunda forma no aparece reflejada en la Figura ya que puede producirse
independientemente del estado en el que se encuentre el hilo.

El atributo \id{Hilo::estado} guarda el estado actual del hilo que puede ser
\id{LISTO,} \id{EJECUCION} o \id{SUSPENDIDO.}
Los m�todos \id{iniciar,} \id{ejecutar,} \id{dormir,} \id{suspender,}
\id{reactivar} y \id{terminar} implementan las transiciones de un estado
a otro, es decir, representan los arcos de la
Figura~\ref{Fig: Estados de un hilo}.
\marginnote{Ver n�cleo/hilo.cpp}

\subsubsection{Cambio de contexto}
Las transiciones a y desde el estado de \emph{Ejecuci�n} implican el cambio
del contexto del hilo, el cual, como ya vimos en la
Secci�n~\ref{Sec: Contexto}, est� formado por los registros del procesador.

Cuando un hilo sale de \emph{Ejecuci�n} se guarda su contexto, es decir, se
copia \id{CPU::contexto} en \id{Hilo::contexto.}
Y al rev�s, cuando un hilo pasa de \emph{Listo} a \emph{Ejecuci�n} se carga
su contexto, es decir, se copia \id{Hilo::contexto} en \id{CPU::contexto.}

\subsection{Las colas de hilos}
Eafitos puede tener un n�mero variable de hilos ejecut�ndose, desde 0 hasta
el valor de la constante \id{N\_HILOS.}
Existe una tabla (un vector) que contiene una entrada para cada hilo, su
definici�n est� en la clase \id{Cola.}
\marginnote{Ver include/cola.h}
El atributo \id{Cola::hilos} es la tabla de hilos, f�jate que est� declarada
como \id{static,} lo que significa que solo existe una tabla en todo el
sistema.
Cada entrada de la tabla puede estar libre u ocupada, una entrada est� libre
si el atributo \id{Hilo::estado} tiene el valor \id{LIBRE;} eso significa que
dicho atributo puede tomar, en realidad, cuatro valores: \id{LIBRE,}
\id{LISTO,} \id{EJECUCION} y \id{SUSPENDIDO.}

La clase \id{Cola,} adem�s de contener la tabla de hilos, implementa una cola
de hilos.
Dicha cola es en realidad una lista circular doblemente enlazada.
El atributo \id{primero} nos identifica el hilo que es cabeza de la cola,
y los atributos \id{anterior} y \id{siguiente} de la clase \id{Hilo} apuntan,
como ya habr�s supuesto, al hilo anterior y siguiente en la cola.
Los m�todos \id{actual,} \id{insertar} y \id{extraer} nos permiten gestionar
la cola.
\marginnote{Ver nucleo/cola.cpp}

Las colas de hilos sirven para mantener ordenados a los hilos cuando se
encuentran en estado \emph{Listo} o \emph{Suspendido}.


\section{Procesos}
\label{Sec: Procesos}

\subsection{Relaci�n con los hilos}
Por un lado en la clase \id{Hilo} se define el atributo \id{proceso} el
cual referencia (es un puntero) al proceso asociado; por otro lado,
la clase \id{Proceso} tiene un atributo llamado \id{uso} que nos indica
el n�mero de hilos asociados al proceso.
\marginnote{Ver include/proceso.h}
Como se puede ver existe una relaci�n 1 a n entre los procesos y los hilos.

Tambi�n se ve que los hilos quedan as� agrupados, perteneciendo a un mismo
grupo aquellos que est�n asociados al mismo proceso.
Pero el proceso no es lo �nico que comparten los hilos del mismo ``grupo''.
Cada hilo tiene dos segmentos de memoria asociados, uno para el c�digo y
los datos est�ticos y otro para la pila.
Todos los hilos de un mismo proceso comparten el mismo segmento de c�digo
y datos est�ticos, pero el segmento de pila no se comparte, es independiente.

\subsection{Vida}
Un proceso nace cuando se ejecuta un programa (ver
Secci�n~\ref{Sec: Ejecucion}); su uso crece cada vez que se crea un hilo
(ver Secci�n~\ref{Sec: Crear hilos}) y decrece cada vez que termina un hilo
(ver Secci�n~\ref{Sec: Terminar hilos}); cuando su uso llega a cero el proceso
termina.

El m�todo \id{Proceso::iniciar} es llamado cada vez que se crea un hilo
y el m�todo \id{Proceso::terminar} cada vez que uno termina
\footnote{Evidentemente, nos referimos siempre a hilos asociados al proceso.}.
\marginnote{Ver nucleo/proceso.cpp}

\subsection{Recursos}
Un hilo puede utilizar los recursos del proceso asociado.
B�sicamente el proceso nos proporciona acceso al sistema de ficheros y
a la entrada/salida, de hecho, la mayor parte de las llamadas al sistema
pasan por la clase \id{Proceso.}

\subsubsection{Unidad y directorio}
Los atributos \id{Proceso::unidad} y \id{Proceso::directorio} indican la
unidad y el directorio actuales, respectivamente.
Ambos par�metros se utilizan en las llamadas al sistema relacionadas con el
sistema de ficheros, la operaci�n requerida se realizar� siempre sobre la
unidad actual, mientras que el directorio actual se utiliza para encontrar
un fichero cuando se utilizan rutas relativas en lugar rutas de absolutas.

Los m�todos \id{cambiarUnidad} y \id{cambiarDirectorio} permiten modificar
estos atributos, implementan las llamadas al sistema equivalentes.

\subsubsection{Sistema de ficheros}
El atributo \id{ficherosLocales} es una tabla que contiene los ficheros
abiertos por el proceso.
Estos ficheros pueden referenciar a un fichero normal (ver
Cap�tulo~\ref{Cap: SF}) o a un controlador de dispositivo de tipo car�cter
(ver Cap�tulo~\ref{Cap: ES}).
Cada uno de estos ficheros se identifica por su posici�n en la tabla (vector).
Los m�todos \id{crear,} \id{abrir,} \id{cerrar,} \id{borrar,} \id{leer,}
\id{escribir} y \id{saltar} implementan las llamadas al sistema equivalentes,
nos permiten trabajar con el sistema de ficheros.
El m�todo \id{ejecutar} sirve para ejecutar programas.

\subsubsection{Entrada/Salida est�ndar}
Como en UNIX, en Eafitos los dispositivos se tratan como si fueran ficheros,
pero la implementaci�n es distinta (m�s sencilla) que la de UNIX.
Esto permite acceder a ellos con los mismos m�todos con los que accedemos a
los ficheros normales.

Adem�s, est�n definidas una entrada y una salida est�ndar, que son los
ficheros locales n�mero 0 y 1 respectivamente.
Espec�ficamente para acceder a la entrada/salida est�ndar, los procesos
disponen de los m�todos \id{leerCaracter} e \id{imprimirCaracter,} los cuales,
como sus nombres indican, leen y escriben un solo car�cter de la entrada y
en la salida est�ndar respectivamente.

\section{El manejador de hilos y procesos}
Este manejador est� definido como la clase \id{ManejadorProcesos.}
\marginnote{Ver include/procesos.h}
F�jate primero en que dicha clase desciende de la clase \id{Cola,} esto
significa que \id{ManejadorProcesos} tiene acceso a la tabla de hilos;
en concreto, este manejador gestiona los hilos que se encuentran en estado
\emph{Listo} y el hilo que est� en \emph{Ejecuci�n}.
Adem�s, contiene la tabla de procesos (atributo \id{procesos}) e implementa
las llamadas al sistema de creaci�n/terminaci�n de hilos y ejecuci�n de
programas.

\subsection{Creaci�n de hilos}
\label{Sec: Crear hilos}
Como ya vimos en la Secci�n~\ref{Sec: Llamadas al sistema} la llamada al
sistema \id{CREAR\_HILO} crea un hilo que es una copia del hilo en
\emph{Ejecuci�n}.
El nuevo hilo estar� asociado al mismo proceso que el actual y ambos
compartir�n la memoria de c�digo y datos (selector de segmento
\id{Contexto::c�digo,} pero la memoria de pila (selector de segmento
\id{Contexto::pila}) ser� diferente.

Esta llamada al sistema est� implementada por el m�todo
\id{ManejadorProcesos::crearHilo,} el cual a su vez hace uso de uno de los
dos m�todos \id{Hilo::iniciar}
\footnote{Aqu�, como en otros sitios, hago uso del polimorfismo.
	Hay dos m�todos iniciar porque hay dos formas de crear un hilo,
	a partir de un hilo ya existente o como uno nuevo e independiente
	(cuando se ejecuta un programa, ver Secci�n~\ref{Sec: Ejecucion}).} 
.
Te aconsejo en este punto que le peches un vistazo al c�digo que est� bien
comentado.

\subsection{Terminaci�n de hilos}
\label{Sec: Terminar hilos}
Existen dos llamadas al sistema relacionadas, una termina el hilo actual y
otra termina un hilo cualquiera (su identificador se pasa como par�metro).

Ambas llamadas son implementadas por el m�todo
\id{terminarHilo,} el cual, lo �nico que hace, adem�s de un par de
comprobaciones, es llamar al m�todo \id{Hilo::terminar} encargado,
como su nombre indica, de terminar el hilo.
Una vez m�s, te invito a que le peches un vistazo al c�digo.

\subsection{El planificador}
Este es el responsable de la pol�tica por la cual un hilo pasa de \emph{Listo}
a \emph{Ejecuci�n} y viceversa.

El planificador est� implementado en el m�todo \id{planificador,} el cual
hace uso de los m�todos \id{Hilo::ejecutar} e \id{Hilo::dormir.}

La pol�tica implementada es \emph{Round-Robin}, es decir, cada cierto tiempo
el hilo en ejecuci�n pasa a ocupar el �ltimo lugar de la cola de hilos listos,
y el primer hilo de la cola pasa a ejecuci�n.
El tiempo se mide en n�mero de instrucciones ejecutadas, la constante
\id{TAJADA} nos da dicho n�mero.


\section{Ejecuci�n de programas}
\label{Sec: Ejecucion}

\subsection{Formatos de fichero ejecutable}
Los programas residen en ficheros con un formato dado, hay que leer dichos
ficheros e interpretarlos.

Eafitos tiene definido un formato de fichero ejecutable concreto llamado 99
\footnote{Por el a�o 1999. Lo mio no es poner nombre a las cosas.}
, el cual est� definido por la clase \id{Ejecutable99.}
Pero adem�s, Eafitos dispone de los recursos necesarios para poder definir
nuevos formatos y a�adirlos f�cilmente al n�cleo.

\subsubsection{Crear nuevos formatos de ejecutables}
Para crear un nuevo formato de ejecutable debes crear una nueva clase que
herede de la clase base \id{Ejecutable.}
\marginnote{Ver include/ejec.h}
El fichero de implementaci�n (\emph{.cpp}) de la nueva clase deber� residir
en el directorio ``nucleo/ejec/''.
Lo �nico que debe implementar la nueva clase es el m�todo \id{ejecutar.}

Una vez hecho esto tan solo queda registrar el nuevo formato en el n�cleo,
esto consiste en ocupar una entrada del vector \id{ejecutables,} el
m�todo \id{ManejadorProcesos::iniciar} se encarga de hacerlo.

Si vas a crear un nuevo formato te aconsejo que estudies el formato que ya
existe, el 99.

\subsubsection{Formato 99}
La clase \id{Ejecutable99,} que desciende de \id{Ejecutable,} es la que
implementa el formato 99.
\marginnote{Ver include/ejec99.h}
En la Figura~\ref{Fig: Ejecutable 99} puedes ver la estructura de un fichero
con este formato.
\begin{figure}
	\includegraphics[width=\textwidth]{figuras/ejec99.eps}
	\caption{Formato de un fichero ejecutable 99.}
	\label{Fig: Ejecutable 99}
\end{figure}

En primer lugar se encuentra la cabecera que contiene informaci�n var�a y
despu�s la imagen del ejecutable que habr� que cargar en memoria.

La cabecera empieza con el n�mero m�gico que identifica al fichero como
ejecutable 99, el n�mero es 2323
\footnote{Dos veces mi edad en el momento de escribir esto.}
.
A continuaci�n viene la direcci�n donde empieza el c�digo, es decir, el
valor con el que hay que inicializar el contador de programa
(\id{Contexto::pc}).
Por �ltimo se encuentra el tama�o m�nimo que debe tener la pila.


\subsection{Ejecuci�n}
El m�todo \id{ManejadorProcesos::ejecutar} es el responsable de realizar
esta operaci�n.
\marginnote{Ver \id{ejecutar} en nucleo/procesos.cpp}
La ejecuci�n de un programa provoca la creaci�n de un hilo y de un proceso,
por ello, lo primero que hace \id{ejecutar} es buscar entradas libres en las
tablas de hilos y de procesos.
Tras esto abre el fichero que se supone contiene la imagen del ejecutable.
A continuaci�n recorre el vector \id{ManejadorProcesos::ejecutables}
llamando cada vez al m�todo \id{Ejecutable::ejecutar,} hasta que el fichero
es reconocido, entonces se crea el hilo y el procesos llamando al m�todo
\id{Hilo::iniciar}.
Si el fichero no se reconoce como ejecutable se genera un error.

\chapter{El n�cleo: Entrada/Salida}
\label{Cap: ES}

Cada tipo de dispositivo de la m�quina virtual debe tener un controlador
asociado en el n�cleo, o de lo contrario no podr� ser utilizado por los
programas de usuario.

Los controladores se pueden clasificar, al igual que los dispositivos, en
dos grupos, de car�cter o de bloque.
Los controladores de tipo car�cter leen y escriben en el dispositivo de forma
secuencial, cada vez un solo car�cter; ejemplos de estos dispositivos son
el teclado y la pantalla.
Los de tipo bloque leen y escriben informaci�n en bloques de datos de longitud
fija y su acceso no es secuencial, es directo; un ejemplo de este tipo son
los discos.

En la Figura~\ref{Fig: Controladores} puedes ver la jerarqu�a de clases
relacionadas con los controladores.
\begin{figure}
	\includegraphics[width=\textwidth]{figuras/controla.eps}
	\caption{Jerarqu�a de clases para los controladores de dispositivos.}
	\label{Fig: Controladores}
\end{figure}


\section{Listas de dispositivos}
Para cada grupo de controladores tenemos sendas listas,
\id{DispositivosCaracter} para los controladores de tipo car�cter y
\id{DispositivosBloque} para los controladores de tipo bloque.
Cada una de estas clases tiene un vector (\id{dispositivos}) cuyas
entradas apuntan a los controladores de dispositivos.
La Figura~\ref{Fig: Dispositivos} ilustra estas estructuras.
\begin{figure}
	\includegraphics[width=\textwidth]{figuras/dispos.eps}
	\caption{Listas de controladores de dispositivos.}
	\label{Fig: Dispositivos}
\end{figure}

\subsubsection{Inicializaci�n}
Ambas clases tienen un m�todo \id{iniciar}
\footnote{Tambi�n tienen un m�todo \id{terminar} pero este no hace nada.}
\marginnote{Ver nucleo/es/bloque.cpp y nucleo/es/caracter.cpp}
que es el encargado de ``cargar'' los controladores de dispositivo en el
atributo \id{dispositivos.}

\subsubsection{Identificaci�n}
Una diferencia importante entre los controladores de tipo car�cter y los
de bloque es la forma en que se identifican.
Mientras que los primeros tienen asociado un nombre los segundos se identifican
por la posici�n que ocupan dentro del vector \id{dispositivos.}
Esto se puede ver con claridad, adem�s de en el m�todo \id{iniciar,} en el
m�todo \id{obtenerControlador} que ambas clases poseen.
\footnote{Inicialmente ambos tipos de dispositivos se identificaban por la
posici�n dentro del atributo \id{dispositivos.}
Modifiqu� el modo de identificaci�n de los controladores de tipo car�cter
cuando implement� la entrada/salida est�ndar como ficheros.
As�, los controladores de tipo car�cter se pueden tratar como ficheros
normales, pero no los controladores de tipo bloque.}


\section{Hilos en estado suspendido}
Como puedes ver en la Figura~\ref{Fig: Controladores} la clase base de toda
la jerarqu�a es \id{Cola,} esto significa que cada controlador de dispositivos
tiene asociada una cola de hilos, la cual representa a los hilos que, en
estado suspendido, est�n esperando para acceder al dispositivo.

De hecho, la mayor parte de la complejidad en la gesti�n de las transiciones
de estados de los hilos de \emph{Ejecuci�n} a \emph{Suspendido} y de
\emph{Suspendido} a \emph{Listo} reside en la clase \id{Controlador.}

Para comprender mejor la forma en que Eafitos simula la entrada/salida, vamos
a estudiar qu� es lo que sucede desde que un hilo realiza una petici�n de
entrada/salida hasta que esta se satisface.

Las peticiones de entrada/salida nacen siempre con una llamada al sistema
y pasan siempre por el sistema de ficheros
\footnote{Esta parte la veremos en el Cap�tulo~\ref{Cap: SF}.}
antes de llegar a uno de los m�todos:
\begin{itemize}
\item \id{ControladorCaracter::leerCaracter}
\item \id{ControladorCaracter::escribirCaracter}
\item \id{ControladorBloque::leer}
\item \id{ControladorBloque::escribir}
\end{itemize}

Entonces empieza el trabajo del subsistema de gesti�n de la entrada/salida.
La petici�n no se satisface inmediatamente sino que provoca la suspensi�n
del hilo que la realiz�, esto est� implementado en el m�todo
\id{nuevaPeticion} de la clase \id{Controlador.}
Una vez el hilo est� en estado \emph{Suspendido} y esperando en la cola del
dispositivo, se procede a registrar la petici�n que realiz�, esto se realiza
con los m�todos \id{Hilo::ponerPeticion} cuya funci�n es almacenar la
informaci�n necesaria en el atributo \id{Hilo::peticion.}
Una vez realizado esto el control regresa al procesador, el cual seguir�
ejecutando instrucciones (de otro hilo que haya pasado a \emph{Ejecuci�n}).

Cada vez que se ejecuta una instrucci�n se produce, como ya vimos en la
Secci�n~\ref{Sec: Reloj}, una llamada al m�todo \id{Nucleo::reloj} el cual
a su vez llama a los m�todos \id{DispositivosCaracter::gestionarColas} y
\id{DispositivosBloque::gestionarColas.}
Estos m�todos recorren las listas de dispositivos llamando a los m�todos
\id{Controlador::planificador} de cada controlador de dispositivo.
Dicho m�todo decrementa el contador \id{Cola::tiempo,} si llega a cero ejecuta
la petici�n pendiente del hilo que est� en la cabeza de la cola (con el
m�todo \id{Controlador::ejecutarPeticion}) y lo pasa de \emph{Suspendido} a
\emph{Listo}.


\section{Crear nuevos controladores}
Si creas nuevos dispositivos en la m�quina virtual tendr�s que escribir
controladores para ellos, si no no podr�s utilizarlos.

Lo primero que debes tener claro es si el dispositivo es de tipo car�cter o
de tipo bloque.
En cualquier caso deber�s crear una clase, la cual deber� heredar de
\id{ControladorCaracter} o \id{ControladorBloque,} seg�n corresponda.

\subsection{De tipo car�cter}
La nueva clase deber� implementar al menos uno de los dos m�todos siguientes:
\begin{itemize}
\item \id{\_leerCaracter} Lee un car�cter del dispositivo y lo devuelve.
\item \id{\_escribirCaracter} Escribe el car�cter que se le pasa en el
	dispositivo.
\end{itemize}

Finalmente deber�s modificar el m�todo \id{DispositivosCaracter::iniciar,}
para que el nuevo controlador sea incluido en la lista de controladores de
tipo car�cter.

\subsection{De tipo bloque}
La nueva clase deber� implementar al menos uno de los dos m�todos siguientes:
\begin{itemize}
\item \id{\_leer} Lee el bloque indicado del dispositivo.
\item \id{\_escribir} Escribe el bloque indicado en el dispositivo.
\end{itemize}

Tambi�n deber�s escribir un constructor que se utilizar� para iniciar el
controlador.

Finalmente deber�s modificar el m�todo \id{iniciar} y/o \id{iniDisco} de la
clase \id{DispositivosBloque}, para que el nuevo controlador sea incluido
en la lista de controladores de tipo bloque.

\chapter{El n�cleo: Sistema de ficheros}
\label{Cap: SF}

Lo normal en la mayor�a de sistemas operativos, o al menos en los sistemas
operativos m�s utilizados, es que estos tan solo soporten un sistema de
ficheros propio y en todo caso alg�n otro heredado de versiones antiguas del
mismo producto.
Otros sistemas operativos, sin embargo, tienen soporte para un mont�n de
sistemas de ficheros.

En principio Eafitos solo entiende un sistema de ficheros, pero dispone de
los mecanismos necesarios para crear f�cilmente nuevos sistemas de ficheros
e incorporarlos al n�cleo.
La Figura~\ref{Fig: SFV} muestra un esquema global del sistema de ficheros.
\begin{figure}
	\includegraphics[width=\textwidth]{figuras/sfv.eps}
	\caption{Esquema del sistema de ficheros}
	\label{Fig: SFV}
\end{figure}

\section{Relaci�n con los procesos}
Una de las posibilidades que ofrecen los procesos a los hilos (ver
Cap�tulo~\ref{Cap: Hilos}) es el acceso al sistema de ficheros a trav�s de
su atributo \id{ficherosLocales} (ver Secci�n~\ref{Sec: Procesos}), en el
dibujo se puede ver que cada entrada de la tabla de ficheros locales apunta
a un fichero de la tabla de ficheros globales (\id{SFV::ficheros}).

Cada entrada de dicha tabla es una instancia de la clase \id{FicheroLocal,}
\marginnote{Ver include/flocal.h}
es a trav�s de esta clase que se accede al sistema de ficheros. 
Pero, como ya vimos en el Cap�tulo~\ref{Cap: Hilos}, un fichero local puede
apuntar a un controlador de dispositivo de tipo car�cter en lugar de a un
fichero normal (o no apuntar a ning�n sitio, si el fichero no est� abierto).
El atributo \id{FicheroLocal::tipo} nos indica a qu� referencia el fichero:
\begin{itemize}
\item \id{LIBRE} El fichero local est� libre.
\item \id{FICHERO} Referencia a un fichero global del sistema de ficheros
	virtual.
	El atributo \id{id} identifica al fichero global
	\footnote{Los arcos de la Figura~\ref{Fig: SFV} que van de la tabla
	de ficheros locales a la de ficheros globales vienen definidos por
	este atributo, \id{FicheroLocal::id}.}
	y el atributo \id{posicion} indica la posici�n dentro del fichero en
	la que realizaremos el pr�ximo acceso.
\item \id{DISPOSITIVO} Referencia a un controlador de dispositivo de tipo
	car�cter.
	El atributo \id{controlador} apunta al controlador de tipo car�cter.
\end{itemize}

Para acceder a estos recursos, la clase \id{FicheroLocal} dispone de los
m�todos \id{crear,} \id{abrir,} \id{cerrar,} \id{leer,} \id{escribir} y
\id{saltar.}

Adem�s de la tabla de ficheros locales los procesos disponen de otros dos
atributos relevantes.
El atributo \id{Proceso::unidad} identifica la unidad actual, es un �ndice
dentro de la tabla de sistemas de ficheros; el atributo
\id{Proceso::directorio} identifica el directorio actual, es un �ndice
dentro de la tabla de ficheros globales.


\section{El sistema de ficheros virtual}
La Figura~\ref{Fig: SFV} muestra dos tablas dentro del sistema de ficheros,
ambas tablas forman parte de lo que se llama sistema de ficheros virtual.
El sistema de ficheros virtual es un interfaz para los sistemas de ficheros
y est� implementado en la clase \id{SFV.}

\subsection{Sistema de ficheros no montado}
El sistema de ficheros de Eafitos no est� montado
\footnote{Convertir el sistema de ficheros de Eafitos en un sistema montado
podr�a ser una buena pr�ctica.}
(como en DOS y al contrario que UNIX), esto lo simplifica significativamente.

De aqu� la necesidad del atributo \id{Proceso::unidad,} que en un sistema de
ficheros montado no ser�a necesario.
En el sistema operativo DOS las unidades se referencian mediante letras (C, D,
...), en Eafitos se usan n�meros (0, 1, ...).

\subsection{Abstracci�n de los sistemas de ficheros}
Igual que los formatos de ficheros ejecutables, los sistemas de ficheros
tambi�n est�n abstra�dos, esto facilita la creaci�n de nuevos sistemas de
ficheros.
La Figura~\ref{Fig: Jerarquia SF} muestra la jerarqu�a de clases existente.
\begin{figure}
	\includegraphics[width=\textwidth]{figuras/sfclases.eps}
	\caption{Jerarqu�a de clases para los sistemas de ficheros.}
	\label{Fig: Jerarquia SF}
\end{figure}

\subsubsection{Creaci�n de nuevos sistemas de ficheros}
Para crear un nuevo sistema de ficheros hay que definir e implementar dos
clases, una de ellas hereda de \id{SF} y la otra hereda de \id{Fichero.}
Si vas a emprender esta tarea f�jate en el sistema de ficheros ya existente.

Adem�s de esto lo �nico que hay que hacer es modificar el m�todo
\id{SFV::iniSF.}
Este m�todo se encarga de inicializar un sistema de ficheros partiendo de un
controlador de dispositivo de tipo bloque.
La inicializaci�n consiste en crear una entrada en la tabla de sistemas de
ficheros (\id{SFV::sf}) y abrir el directorio ra�z.
El fragmento de c�digo a modificar es:
\begin{quote}
\begin{verbatim}
try {
        sf[n] = new EafitSF(cB);
}
catch(...) {
        throw "sistema de ficheros desconocido";
}
\end {verbatim}
\end{quote}
y el c�digo modificado: 
\begin{quote}
\begin{verbatim}
try {
        sf[n] = new EafitSF(cB);
}
catch(...) {
        try {
                sf[n] = new NuevoSF(cB);
        }
        catch(...) {
                throw "sistema de ficheros desconocido";
        }
}
\end {verbatim}
\end{quote}

Donde \verb|NuevoSF| es la clase del nuevo sistema de ficheros que desciende
de \id{SF.}
Si el dispositivo controlado por el controlador \id{cB} no contiene un sistema
de ficheros tipo Eafit entonces prueba con el sistema de ficheros Nuevo; de
este modo se podr�an crear cuantos sistemas de ficheros se quisiera, sin
l�mites.

\subsubsection{La clase \id{SF}}
El atributo \id{SFV::sf} es un vector de punteros a la clase \id{SF.}
De este modo, cada puntero puede apuntar a una instancia de una clase distinta,
siempre y cuando dicha clase herede de \id{SF.}

El objetivo de esta clase, o mejor dicho, de sus descendientes, es el de
tratar directamente con el sistema de ficheros; esta es la clase que entiende
c�mo esta organizada la informaci�n en el disco.

Un sistema de ficheros no es m�s que un contenedor de ficheros.
Dentro de cada sistema de ficheros los ficheros deben quedar identificados
un�vocamente por un n�mero, y el directorio ra�z debe ser siempre el fichero
n�mero 0.
Por ejemplo, en un sistema de ficheros tipo UNIX (como el que tiene Eafitos)
los ficheros se identifican por su inodo, mientras que en el sistema FAT de
DOS los ficheros se identifican por el n�mero de su primer bloque; en cualquier
caso, siempre se un n�mero.

La clase \id{SF} tiene varios atributos y m�todos que son comunes para
cualquier clase que descienda de ella.
El atributo \id{dispositivo} apunta al controlador de dispositivo asociado,
el cual proporciona, como ya sabes, un acceso directo por bloques.
Los atributos \id{tamBloque} y \id{numBloques} contienen el tama�o del bloque
y el n�mero de bloques del dispositivo.
Como ya vimos anteriormente, cuando se inicializa un sistema de ficheros se
abre su directorio ra�z; el atributo \id{dirRaiz} identifica el fichero global
que representa al directorio ra�z abierto.
Adem�s de estos atributos la clase \id{SF} tiene varios m�todos obvios.

Si creas un nuevo sistema de ficheros tendr�s que crear una clase que herede
de \id{SF.}
Esa nueva clase deber� implementar el m�todo \id{abrir,} el cual abre el
fichero que se le indica (mediante el n�mero que lo identifica).
Tambi�n deber�s implementar dos constructores, uno que sirva para inicializar
el sistema de ficheros y que tan solo reciba como par�metro el controlador de
bloque, y otro que sirva para dar formato a un disco; probablemente tambi�n
tendr� que escribir un destructor.
Y por supuesto tendr�s que implementar todo lo que el sistema de ficheros
concreto que estas creando requiera.

\subsubsection{La clase \id{Fichero}}
De forma similar, el atributo \id{SFV::ficheros} es un vector de punteros
a la clase \id{Fichero.}
Esta clase guarda informaci�n sobre cada fichero abierto del sistema.
Su atributo \id{uso} indica cuantos elementos est�n utilizando este fichero,
el atributo \id{posicion} indica cual es la posici�n en la que se har� el
pr�ximo acceso, el atributo \id{tamBloque} contiene el tama�o del bloque del
sistema de ficheros y \id{buffer} no es m�s que un espacio de memoria que
usaremos para leer y escribir bloques en el sistema de ficheros.

Adem�s de estos atributos, \id{Fichero} tiene definidos un conjunto de m�todos
que deber�n ser implementados en las clases descendientes.
Los m�todos \id{es,} \id{obtenerTamano,} \id{finalFichero,} \id{abrir,}
\id{cerrar,} \id{borrar,} \id{leer,} \id{escribir} y \id{saltar} se deben
implementar siempre porque son generales para cualquier fichero.
Sin embargo existen otros m�todos que son espec�ficos para directorios y por
lo tanto solo deber�n ser implementados por estos.
Estos m�todos y su significado se describen a continuaci�n:
\begin{itemize}
\item \id{idFichero} Dentro de un directorio un fichero se identifica
	un�vocamente por su nombre; este m�todo debe devolver, a partir del
	nombre del fichero, el n�mero que lo identifica dentro del sistema
	de ficheros.
\item \id{existeFichero} Devuelve cierto si existe y falso si no.
\item \id{crearFichero} Crea y abre el fichero.
\item \id{borrarFichero} Borra el fichero dado, si existe claro.
\item \id{infoFichero} Rellena la estructura \id{InfoFichero,} esto se utiliza
	para listar el contenido de un directorio.
\end{itemize}

\subsection{Interfaz}
Vamos a estudiar ahora con cierto detalle la clase \id{SFV.}
Esta clase tan solo tiene dos atributos que ya hemos visto antes, la tabla de
ficheros globales, \id{ficheros,} y la tabla de sistemas de ficheros, \id{sf.}

Los m�todos \id{iniciar} y \id{terminar} son los encargados, como sus nombres
indican, de iniciar y terminar el sistema de ficheros.
Estos m�todos son llamados desde el n�cleo.
En concreto el m�todo \id{iniciar} lo que hace es inicializar la tabla de
sistemas de ficheros, para cada disco formateado crea inicializa una entrada,
para ello se sirve del m�todo \id{iniSF;} adem�s, para cada sistema de ficheros
abre su directorio ra�z.

El resto de m�todos son usados para acceder a los ficheros e implementan, en
parte, las llamadas al sistema equivalentes.
Estos m�todos son \id{crear,} \id{abrir,} \id{cerrar,} \id{borrar,}
\id{leer,} \id{escribir,} \id{saltar} e \id{infoFichero;} despu�s de haber
le�do la secci�n anterior resulta obvio saber qu� es lo que hacen.

\subsubsection{B�squeda de un fichero}
Algunos de los m�todos de \id{SFV} (\id{crear,} \id{abrir} y \id{borrar})
reciben como par�metro, entre otros, el nombre del fichero sobre el que
realizar la operaci�n.
A partir de este nombre hay que, en primer lugar, obtener el identificador
del fichero.
El nombre puede ser una ruta absoluta (ejemplo, \emph{/bin/fich}) o una ruta
relativa al directorio actual (ejemplo, \emph{../fich}).
Lo que vamos a ver ahora, sin llegar al detalle, es el algoritmo que permite
obtener el identificador del fichero a partir de su nombre.

Antes de seguir adelante te aconsejo que cojas el c�digo y leas lo que sigue
comprobando la implementaci�n de alguno de estos m�todos.
La explicaci�n que viene a continuaci�n se refiere al m�todo \id{abrir,} la
implementaci�n de los otros m�todos es muy parecida.

Lo primero que se debe determinar es si la ruta es absoluta o relativa (esto
se hace en funci�n del primer car�cter del nombre), si la ruta es absoluta
se empezar� buscando desde el directorio ra�z y si es relativa se empezar�
desde el directorio actual (ambos directorios est�n abiertos y son conocidos).
A continuaci�n nos introducimos en un bucle, con cada iteraci�n del bucle
buscamos el siguiente subdirectorio, si el subdirectorio ya est� abierto
aprovechamos su entrada en la tabla de ficheros globales y si no lo est� lo
abrimos temporalmente con \id{SF::abrir;} si el subdirectorio no existe se
aborta la operaci�n y se genera un error.
La ejecuci�n del bucle termina cuando llegamos al �ltimo elemento del nombre
del fichero, entonces, si todo ha salido bien habremos obtenido su
identificador y podremos abrirlo.


\section{El sistema de ficheros Eafit}
Se trata de un sistema de ficheros tipo UNIX.

\subsection{Estructura en disco}
En la Figura~\ref{Fig: Eafit SF} puedes ver c�mo divide el disco este sistema
de ficheros.
La clase \id{EafitSF,} que desciende de \id{SF,} es la encargada de gestionar
el sistema de ficheros a este nivel.
\begin{figure}
	\includegraphics[width=\textwidth]{figuras/eafitsf.eps}
	\caption{Sistema de ficheros Eafit}
	\label{Fig: Eafit SF}
\end{figure}

\subsubsection{El superbloque}
El primer bloque del disco est� ocupado por el superbloque.
Este contiene cierta informaci�n global empezando por un n�mero que identifica
al sistema de ficheros Eafit, este n�mero se escribe cuando se da formato a
un disco y se busca al inicializar (si no lo encuentra no inicializa el
sistema de ficheros).

Adem�s del n�mero de identificaci�n el superbloque contiene 8 datos m�s que
identifican los n�meros de bloque donde empiezan el resto de zonas del sistema
de ficheros (mapas de inodos y de bloques de datos, inodos y datos), y el
n�mero de bloques que ocupan.
Como puedes ver hay informaci�n redundante, esta redundancia evita tener que
hacer c�lculos al inicializar.
El superbloque est� definido en la clase \id{SuperbloqueESF.}

\subsubsection{Los mapas de inodos y de datos}
Los mapas de inodos y de datos nos indican si un inodo (o un bloque de datos)
est� libre u ocupado.
Para hacer esto solamente es necesario un bit pero este sistema de ficheros
gasta un byte (para simplificar), algo que podr�a cambiar en versiones
posteriores.

\subsubsection{Los inodos}
Cada fichero tiene un inodo que recoge cierta informaci�n sobre el fichero.
Cada inodo ocupa 64 bytes.
Los inodos est�n definidos en la clase \id{InodoESF,} cuyos campos son:
\begin{itemize}
\item \id{tamano} El tama�o del fichero.
\item \id{tipo} Puede ser \id{NORMAL} o \id{DIRECTORIO.}
\item \id{reservado} 27 bytes para uso futuro.
\item \id{bloquesDirectos} Un vector de siete entradas, cada una es un n�mero
	de bloque, cada uno de los cuales almacena los datos del fichero.
\item \id{bloquesIndirectos} Es el n�mero de un bloque, el cual contiene a su
	vez m�s n�meros de bloques donde podremos encontrar m�s datos del
	fichero.
\end{itemize}

\subsubsection{Datos}
Aqu� es donde se almacena el contenido de los ficheros.

\subsection{Ficheros}
Los ficheros est�n implementados en la clase \id{EFichero,} b�sicamente esta
clase lo que hace es proporcionar algunos m�todos para acceder a los ficheros
(leer, escribir,...).

Sin duda lo m�s complicado de esta parte es la lectura y escritura.
Los m�todos \id{leer} y \id{escribir} hacen uso de los m�todos
\id{obtenerBloque} y \id{reservarBloque,}.

El m�todo \id{obtenerNBloque} devuelve el n�mero de bloque del sistema de
ficheros que corresponde al atributo \id{poscion,} si la posici�n est� fuera
del fichero se genera un error.
El m�todo \id{reservarBloque} es similar a \id{obtenerNBloque} excepto en que
si la posici�n est� fuera del fichero en lugar de generar un error aumenta el
tama�o del fichero y si es necesario reserva un nuevo bloque.
Como habr�s imaginado el primer m�todo se usa para leer y el segundo para
escribir.



\subsubsection{Directorios}
Los directorios est�n implementados por la clase \id{EDirectorio,} que
desciende de la clase \id{EFichero} y por lo tanto hereda su funcionalidad.

El contenido de los directorios est� dividido en entradas de longitud fija
(32 bytes) definidas en la clase \id{EEntradaDirectorio.}
Los campos de cada entrada son:
\begin{itemize}
\item \id{nInodo} Si tiene un valor negativo significa que la entrada est�
	libre, si el valor es positivo identifica al n�mero de inodo del
	fichero.
\item \id{nombre} El nombre del fichero, m�ximo 20 caracteres.
\item \id{reservado} Ocho bytes para uso futuro.
\end{itemize}

Todos los directorios excepto el directorio ra�z tienen siempre una entrada
ocupada (\emph{..}), esta entrada hace referencia al directorio padre. 


\section{Recorrido de una llamada al sistema de ficheros}
Ya hemos visto todo el sistema de ficheros, esta secci�n lo que pretende es
dar otro enfoque al problema para mejorar la comprensi�n del sistema.
Para ello vamos a ver las clases por las que pasa una llamada al sistema
de ficheros.
Por ejemplo una llamada para leer uno o m�s bytes de un fichero ya abierto.

La llamada nace de la instrucci�n \textbf{ser\_sis}, cuando el procesador la
encuentra realiza la pertinente llamada al n�cleo, al m�todo
\id{Nucleo::llamada.}
El n�cleo recibe la llamada, la decodifica y llama al m�todo que corresponda
del proceso actual, en este caso \id{Proceso::leer.}
El proceso repite la llamada al sistema de ficheros virtual, \id{SFV::leer,}
y este llama al fichero, \id{Fichero::leer.}
A partir de aqu� la operaci�n se resuelve con una o m�s llamadas a la clase
encargada de gestionar el sistema de ficheros concreto (en nuestro caso
\id{EafitSF}), esto desencadenar� a su vez llamadas al controlador de
dispoositivo y de ah� al dispositivo.
Una vez la operaci�n se haya ejecutado se recorrer� el camino inverso: del
fichero al sistema de ficheros virtual, despu�s al proceso, al n�cleo y
finalmente el control regresa al procesador.

La descripci�n anterior ha sido realizada sin tener en cuenta que las llamadas
al subsistema de entrada/salida provocan, como ya vimos en anteriores
cap�tulos, la suspensi�n del hilo que la realiz�.
Por lo tanto la realidad es significativamente m�s compleja, de todos modos
espero que esta descripci�n haya dado una idea m�s coherente del sistema
de ficheros y de lo que sucede cuando se produce una llamada.
Ahora tan solo queda estudiar el c�digo fuente, modificarlo a tu gusto, y si
algo no te ha quedado claro... volver a leer esta documentaci�n (o preguntarme
a mi $\smile$).

\chapter{El entorno}

La ejecuci�n comienza en la funci�n \id{main}.
\marginnote{Ver eafitos2.cpp}
Primero inicia la m�quina virtual y el n�cleo, despu�s cede el control al
entorno ejecutando el m�todo \id{Entorno::iniciar} y finalmente termina
el n�cleo y la m�quina virtual.

El entorno no es m�s que el sistema de men�s que aparece cuando se ejecuta
Eafitos, el cual permite crear y formatear discos e iniciar el int�rprete
de comandos.
Est� definido e implementado por la clase \id{Entorno} en los ficheros
\emph{include/entorno.h} y \emph{entorno/entorno.cpp}, es muy simple, as� que
no lo voy a comentar.


\section{El int�rprete de comandos}
El int�rprete de comandos est� definido e implementado por la clase.
\id{InterpreteComandos}.
\marginnote{Ver include/ic.h y entorno/ic.cpp}
Consta de un s�lo m�todo llamado \id{iniciar}, que es muy sencillo.
B�sicamente se trata de un bucle infinito, al principio del cual se lee una
l�nea del teclado para analizarla a continuaci�n.
En funci�n del comando que haya escrito el usuario se realizar� una acci�n u
otra, o ninguna si el comando no est� definido.
Del bucle se sale cuando el usuario escribe el comando \emph{salir}.

\chapter{El compilador}


\section{Introducci�n}

El compilador es m�s exactamente un ensamblador, ya que lo que hace es
traducir instrucciones del lenguaje ensamblador
(ver Secci�n~\ref{Sec: Arquitectura}) a las instrucciones del c�digo m�quina
que el procesador entiende.
Adem�s, el resultado es un fichero con un formato dado, en concreto el
formato 99.

El ensamblador est� implementado por la clase \id{Ensamblador}.
\marginnote{Ver include/ensambla.h}
En las siguientes secciones estudiaremos por separado cada una de las partes
de que consta: analizador l�xico, analizador sint�ctico, analizador sem�ntico
y generador de c�digo.

\subsection{Dos pasadas}
El lenguaje permite utilizar un identificador en una instrucci�n y declararlo
en un lugar posterior.
Esto es necesario para poder hacer saltos hacia delante, pero complica el
compilador.
La soluci�n adoptada consiste en dar dos pasadas al c�digo fuente.

En la primera se genera cierta informaci�n interna que se utilizar� en la
segunda.
En la segunda pasada se genera el c�digo.
En la Figura~\ref{Fig: Ensamblador} puedes ver un dibujo que muestra la
relaci�n entre las distintas partes funcionales y la/s pasada/s en las que
intervienen.
\begin{figure}
	\includegraphics[width=\textwidth]{figuras/ensambla.eps}
	\caption{Esquema del ensamblador}
	\label{Fig: Ensamblador}
\end{figure}

En el dibujo, aparecen duplicadas las distintas partes del ensamblador,
eso no significa que hayan dos de cada, est� dibujado as� para ver mejor
las pasadas.

Como se puede observar, el analizador l�xico genera \emph{tokens} a medida
que el analizador sint�ctico se los pide. 
Tambi�n se ve que en ambas pasadas intervienen los analizadores l�xico y
sint�ctico, los cuales realizan exactamente la misma tarea en ambas ocasiones.
Por el contrario, el analizador sem�ntico realiza operaciones distintas en
cada pasada, lo veremos con detalle m�s adelante.
El generador de c�digo solo interviene en la �ltima pasada, cuando ya est�
disponible toda la informaci�n que necesita.
El atributo \id{pasada} indica en qu� pasada se encuentra el an�lisis.

El an�lisis empieza en el m�todo p�blico \id{analizar}, que viene a ser el
\emph{main} del compilador.
\marginnote{Ver entorno/ensambla.cpp}
Es recomendable echarle un vistazo al c�digo, ya que, al menos en este
caso, est� bien comentado.

\subsection{Gesti�n de errores}
La gesti�n de errores de este compilador es extremadamente sencilla.
Cuando se detecta un error, da igual de qu� tipo sea, se genera el error
y se detiene la compilaci�n.
Es decir, no existe recuperaci�n de errores, por lo tanto solo se puede
detectar uno cada vez.
Adem�s, si el error se produjo en la segunda pasada, se borra el c�digo
objeto generado.

\section{Analizador L�xico}
\subsection{Especificaci�n l�xica}
\subsubsection{Blancos}
Consideramos blancos al espacio, los tabuladores horizontal y vertical, el
retorno de carro y el ``form-feed''.
Dichos caracteres se ignoran, no devuelven ning�n token ni tienen ninguna
acci�n asociada.
Su expresi�n regular es:
\begin{quote}
	\verb|\ \r\t\v\f|
\end{quote}

\subsubsection{Comentarios}
Al igual que los blancos, los comentarios tambi�n se ignoran.
Comienzan por un punto y coma y terminan con una nueva l�nea, su expresi�n
regular es:
\begin{quote}
	\verb|;.|$^*\backslash$ n
\end{quote}
La nueva l�nea tan solo indica cuando termina el comentario, no est�
incluida en la expresi�n.

\subsubsection{Nueva l�nea}
La nueva l�nea se emplea como un separador en el nivel sint�ctico, por ello
se devuelve el token \id{NUEVA\_LINEA}.
Comienza siempre por una nueva l�nea y traga todos los blancos, comentarios y
nuevas l�neas que le sigan, hasta que encuentra un car�cter distinto.
Su expresi�n regular es:
\begin{quote}
	\verb!\n([\ \n\r\t\v\f]|(;.!$^*$\verb!\n))!$^*$
\end{quote}

\subsubsection{Final de fichero}
Como ya habr�s imaginado, este token (\id{FINAL\_FICHERO}) se emite cuando
se termina de leer el fichero.

\subsubsection{Palabras clave}
La Tabla~\ref{Tab: Palabras Clave} muestra la lista de palabras clave del
ensamblador.
\begin{table}
\center
\begin{tabular}{lll|lll}
\emph{lexema}	&	\emph{token}	&	\emph{valor}	&
\emph{lexema}	&	\emph{token}	&	\emph{valor}	\\
\hline
\er{datos}	&	\id{DATOS}	&			&
\er{codigo}	&	\id{CODIGO}	&			\\
\er{sumar}	&	\id{I\_RRR}	&	\id{SUMAR}	&
\er{restar}	&	\id{I\_RRR}	&	\id{RESTAR}	\\
\er{and}	&	\id{I\_RRR}	&	\id{AND}	&
\er{or}		&	\id{I\_RRR}	&	\id{OR}		\\
\er{copiar}	&	\id{I\_RR}	&	\id{COPIAR}	&
\er{not}	&	\id{I\_RR}	&	\id{NOT}	\\
\er{cargar32}	&	\id{I\_RR}	&	\id{CARGAR32}	&
\er{guardar32}	&	\id{I\_RR}	&	\id{GUARDAR32}	\\
\er{cargar8}	&	\id{I\_RR}	&	\id{CARGAR8}	&
\er{guardar8}	&	\id{I\_RR}	&	\id{GUARDAR8}	\\
\er{cargar\_i}	&	\id{I\_RI}	&	\id{CARGAR\_I}	&
\er{guardar\_i}	&	\id{I\_RI}	&	\id{GUARDAR\_I}	\\
\er{apilar}	&	\id{I\_R}	&	\id{APILAR}	&
\er{desapilar}	&	\id{I\_R}	&	\id{DESAPILAR}	\\
\er{saltar}	&	\id{I\_I}	&	\id{SALTAR}	&
\er{saltar0}	&	\id{I\_RI}	&	\id{SALTAR0}	\\
\er{saltarp}	&	\id{I\_RI}	&	\id{SALTARP}	&
\er{saltarn}	&	\id{I\_RI}	&	\id{SALTARN}	\\
\er{nop}	&	\id{I\_}	&	\id{NOP}	&
\er{ser\_sis}	&	\id{I\_}	&	\id{SER\_SIS}	\\
\end{tabular}
	\caption{Palabras Clave.}
	\label{Tab: Palabras Clave}
\end{table}

Las palabras clave, igual que los identificadores (ver m�s adelante), no son
sensibles a may�sculas/min�sculas.
Es lo mismo escribir \er{datos} que \er{DaToS}.

Como se puede observar en la tabla, la mayor�a de las palabras claves son
instrucciones del lenguaje ensamblador, las cuales se agrupan (mediante el
token) seg�n su formato, es decir, seg�n su n�mero y tipo de operandos.
Sin operandos, con uno, dos o tres registros, con un dato inmediato
o con un registro y un dato inmediato.
Dentro de cada grupo el \emph{valor} identifica de qu� instrucci�n se trata;
\emph{valor} almacena el c�digo de instrucci�n que el procesador entiende.


\subsubsection{Identificadores}
Los identificadores se utilizan para especificar variables y posiciones
dentro del c�digo.
Cuando se detecta un identificador se emite el token \id{ID}.
Su expresi�n regular, exceptuando las palabras clave, es:
\begin{quote}
	\verb [a-zA-Z][a-zA-Z0-9_] $^*$
\end{quote}

\subsubsection{Literales num�ricos}
Se utilizan para inicializar las variables y para los datos inmediatos.
El token que se emite es \id{LITERAL\_NUMERICO}.
Su expresi�n regular es:
\begin{quote}
	\verb #[0-9] $^+$
\end{quote}

En el atributo \emph{valor} se almacena el valor num�rico del literal.

\subsubsection{Literales cadena}
Se utilizan para inicializar las variables.
El token que se emite es \id{CADENA}.
Su expresi�n regular es:
\begin{quote}
	\verb ". $^*$\verb|"|
\end{quote}

En el atributo \emph{lexema} se almacena la cadena, sin las comillas.

\subsubsection{Registros}
Se utilizan para especificar los operandos de tipo registro.
El token que se emite es \id{REGISTRO}.
Su expresi�n regular es:
\begin{quote}
	\verb|@[0-1][0-5]?|
\end{quote}

En el atributo \emph{valor} se almacena el n�mero del registro.

\subsubsection{Coma}
Se utiliza para separar los operandos.
El token que se emite es \id{COMA}.
Su expresi�n regular es:
\begin{quote}
	\er{,} 
\end{quote}


\subsection{Aut�mata finito determinista}
Lo encontrar�s en la Figura~\ref{Fig: AFD}.

\begin{figure}
	\includegraphics[width=\textwidth]{figuras/afd.eps}
	\caption{Aut�mata Finito Determinista}
	\label{Fig: AFD}
\end{figure}

\subsection{Cuestiones de implementaci�n}
El tipo enumerado \id{Tokens} define los tokens que existen.
\marginnote{Ver include/ensambla.h}
El m�todo \id{siguienteToken} implementa el analizador l�xico.
Este m�todo devuelve un token cada vez que se le llama.
Adem�s, el �ltimo token le�do se guarda en el atributo \id{token}, su lexema
en \id{lexema} y, dependiendo del token, \id{valor} puede almacenar m�s
informaci�n acerca del token.

Para estudiar este m�todo lo mejor es leer el c�digo con el dibujo del AFD
(Figura~\ref{Fig: AFD}) al lado, ya que lo �nico que hace es implementar dicho
aut�mata.
B�sicamente es un bucle infinito del que se sale cuando se encuentra un token.
Los arcos del aut�mata quedan representados por estructuras condicionales,
primero se lee un car�cter del c�digo fuente y dependiendo de qu� car�cter
se trate se pasa a analizar un subgrafo u otro.
Mejor dale un vistazo al c�digo.

\subsubsection{Palabras reservadas}
Una palabra reservada se define mediante la clase \id{PalabraClave}.
Cada palabra reservada tiene un \emph{token} asociado, un lexema (que debe
estar siempre en may�sculas) y opcionalmente un valor.
La variable \id{palabrasClave} es un vector con una entrada para cada
palabra reservada.
\marginnote{Ver entorno/ensambla.cpp}
La constante \id{N\_PALABRAS\_CLAVE} indica el n�mero de palabras reservadas
que existen. 

\subsubsection{N�mero de l�nea}
Adem�s, el m�todo \id{siguienteToken} lleve la cuenta del n�mero de l�nea que
se est� analizando.
El n�mero de l�nea actual se guarda en el atributo \id{linea}.
As�, cuando se produzca un error, sabremos en qu� l�nea del c�digo fuente se
produjo.

\section{Analizador sint�ctico}
\subsection{Especificaci�n sint�ctica}
En la Tabla~\ref{Tab: Sintaxis} encontrar�s la especificaci�n sint�ctica.
Los s�mbolos terminales est�n en negrita y empiezan por min�scula, se
corresponden con los \emph{tokens} del nivel l�xico (\te{nl} es una
abreviatura de \te{nueva\_linea}).
Los s�mbolos no terminales empiezan por may�scula.

\begin{table}
\begin{tabular}{lll}
\nt{Programa} & $\rightarrow$
	 & \te{nl}\te{datos}\te{nl}\nt{Datos}\te{codigo}\te{nl}\nt{Codigo}
	   \te{final\_fichero} $\mid$\\
	&& \te{datos}\te{nl}\nt{Datos}\te{codigo}\te{nl}\nt{Codigo}
	   \te{final\_fichero} $\mid$\\
	&& \te{nl}\te{codigo}\te{nl}\nt{Codigo}\te{final\_fichero} $\mid$\\
	&& \te{codigo}\te{nl}\nt{Codigo}\te{final\_fichero}\\
\nt{Datos} & $\rightarrow$
	 & \nt{LineaDatos}\te{nl}\nt{Datos} $\mid$\\
	&& $\lambda$\\
\nt{LineaDatos} & $\rightarrow$
	 & \te{id}\nt{Constante}\\
\nt{Constante} & $\rightarrow$
	 & \te{cadena} $\mid$ \te{literal\_num�rico}\\
\nt{Codigo} & $\rightarrow$
	 & \nt{LineaCodigo}\te{nl}\nt{Codigo} $\mid$\\
	&& \nt{LineaCodigo}\te{nl} $\mid$\\
	&& \nt{LineaCodigo}\\
\nt{LineaCodigo} & $\rightarrow$
	 & \te{id}\nt{Instruccion} $\mid$\\
	 & \te{id} $\mid$\\
	&& \nt{Instruccion}\\
\nt{Instrucci�n} & $\rightarrow$
	 & \te{i\_} $\mid$\\
	&& \te{i\_r}\te{registro} $\mid$\\
	&& \te{i\_rr}\te{registro}\te{coma}\te{registro} $\mid$\\
	&& \te{i\_rrr}\te{registro}\te{coma}\te{registro}\te{coma}
	   \te{registro} $\mid$\\
	&& \te{i\_i}\te{literal\_num�rico} $\mid$\\
	&& \te{i\_i}\te{id} $\mid$\\
	&& \te{i\_ri}\te{registro}\te{coma}\te{literal\_num�rico}\\
	&& \te{i\_ri}\te{registro}\te{coma}\te{id}\\
\end{tabular}
	\caption{Especificaci�n sint�ctica}
	\label{Tab: Sintaxis}
\end{table}

Como puedes observar, esta no es una gram�tica LL(1).
El lenguaje es tan sencillo que resulta f�cil de implementar directamente la
gram�tica incontextual.
 
\subsection{Cuestiones de implementaci�n}
Para cada s�mbolo no terminal existe un m�todo, con el mismo nombre, que lo
implementa.
B�sicamente, lo que hacen estos m�todos es recorrer la parte derecha de la
producci�n.

Si a continuaci�n hay un s�mbolo terminal, se lee el siguiente token y si
es el esperado se sigue con el an�lisis, si es otro se produce un error.
Si a continuaci�n hay un s�mbolo no terminal se llama al m�todo que lo
implementa.

Como es una gram�tica incontextual puede suceder que en un momento dado haya
m�s de una posibilidad (por ejemplo, dos s�mbolos no terminales distintos),
entonces se lee un token, y dependiendo de qu� token se trate se hace una
cosa u otra.

Estos m�todos, adem�s de implementar el analizador sint�ctico, implementan
el analizador sem�ntico y el generador de c�digo.

\section{Analizador sem�ntico}
\subsection{Especificaci�n sem�ntica}
Las comprobaciones sem�nticas que realiza el compilador son:
\begin{enumerate}
\item No se puede declarar un identificador m�s de una vez.
\item No puede referenciarse ning�n identificador que no est� declarado; es
  indiferente que la declaraci�n sea anterior o posterior a su referencia.
\end{enumerate}

No se distingue entre identificadores que representan variables o posiciones
dentro del c�digo.

\subsection{Cuestiones de implementaci�n}
El analizador sem�ntico interviene en las dos pasadas que da el ensamblador
al c�digo fuente, pero hace una cosa distinta en cada una de ellas.

En la primera pasada recoge informaci�n acerca de los identificadores y
comprueba la regla sem�ntica n�mero 1.
Dicha informaci�n se guarda en el vector \id{ids}, cuyas entradas son
instancias de la clase \id{Id}.
\marginnote{Ver include/ensambla.h}
En concreto, la informaci�n que se almacena es el lexema del identificador
y la direcci�n de memoria que referencia. 
El atributo \id{direccion} indica la direcci�n actual de memoria, este
atributo es incrementado por el analizador sem�ntico cada vez que encuentra
una instrucci�n o la declaraci�n de una variable; se utiliza para saber qu�
direcci�n asociar a los identificadores.

El m�todo \id{nuevoId} es el que se encarga de guardar los identificadores en
el vector y de incrementar el atributo \id{nIds}, el cual guarda el n�mero de
identificadores almacenados; tambi�n es este m�todo el que realiza la
comprobaci�n sem�ntica n�mero 1.

En la segunda pasada se comprueba la regla sem�ntica n�mero 2.
Dicha comprobaci�n la realiza el m�todo \id{obtenerDir}, el cual devuelve la
direcci�n asociada a un identificador, o genera un error si el identificador
no existe.
Este m�todo es utilizado por el generador de c�digo.

\section{Generador de c�digo}
La generaci�n de c�digo es muy sencilla ya que se trata de un lenguaje
ensamblador.

\subsubsection{Cabecera}
El c�digo generado sigue el formato de fichero ejecutable 99, eso significa
que lo primero que se debe escribir en el fichero es la cabecera.
Esa informaci�n est� formada por el ``n�mero m�gico'' que autentifica al
fichero, la direcci�n de inicio del programa, contenida en el atributo
\id{dirInicio}, y el tama�o de la pila que es siempre de 128.

\subsubsection{�rea de datos}
Despu�s de la cabecera comienza la imagen que se cargar� en memoria.
Esta empieza por el �rea de datos, los cuales son inicializados por el
generador de c�digo.

\subsubsection{C�digo}
Y despu�s de los datos, el c�digo.
Cada instrucci�n en ensamblador se traduce directamente a c�digo m�quina.
Al final se a�aden siempre tres instrucciones que lo que hacen es terminar
la ejecuci�n del hilo, esas instrucciones en ensamblador son:
\begin{quote}
\begin{verbatim}
	cargar_i @0, #3
	apilar @0
	ser_sis
\end{verbatim}
\end{quote}


\part{Ap�ndices}
\begin{appendix}
\chapter{C�digo fuente}
\label{Codigo fuente}

\section{Ficheros de cabecera}
\begin{codigo}
\verbatimtabinput[6]{../../programa/include/mv.h}
\newpage
\verbatimtabinput[6]{../../programa/include/memfis.h}
\newpage
\verbatimtabinput[6]{../../programa/include/mmu.h}
\newpage
\verbatimtabinput[6]{../../programa/include/contexto.h}
\newpage
\verbatimtabinput[6]{../../programa/include/cpu.h}
\newpage
\verbatimtabinput[6]{../../programa/include/teclado.h}
\newpage
\verbatimtabinput[6]{../../programa/include/pantalla.h}
\newpage
\verbatimtabinput[6]{../../programa/include/disco.h}
\newpage

\verbatimtabinput[6]{../../programa/include/nucleo.h}
\newpage
\verbatimtabinput[6]{../../programa/include/memoria.h}
\newpage
\verbatimtabinput[6]{../../programa/include/hilo.h}
\newpage
\verbatimtabinput[6]{../../programa/include/proceso.h}
\newpage
\verbatimtabinput[6]{../../programa/include/cola.h}
\newpage
\verbatimtabinput[6]{../../programa/include/procesos.h}
\newpage

\verbatimtabinput[6]{../../programa/include/peticion.h}
\newpage
\verbatimtabinput[6]{../../programa/include/controla.h}
\newpage
\verbatimtabinput[6]{../../programa/include/ccar.h}
\newpage
\verbatimtabinput[6]{../../programa/include/cteclado.h}
\newpage
\verbatimtabinput[6]{../../programa/include/cpantall.h}
\newpage
\verbatimtabinput[6]{../../programa/include/cbloque.h}
\newpage
\verbatimtabinput[6]{../../programa/include/cdisco.h}
\newpage
\verbatimtabinput[6]{../../programa/include/caracter.h}
\newpage
\verbatimtabinput[6]{../../programa/include/bloque.h}
\newpage

\verbatimtabinput[6]{../../programa/include/sf.h}
\newpage
\verbatimtabinput[6]{../../programa/include/flocal.h}
\newpage
\verbatimtabinput[6]{../../programa/include/eafitsf.h}
\newpage

\verbatimtabinput[6]{../../programa/include/ejec.h}
\newpage
\verbatimtabinput[6]{../../programa/include/ejec99.h}
\newpage

\verbatimtabinput[6]{../../programa/include/entorno.h}
\newpage
\verbatimtabinput[6]{../../programa/include/ic.h}
\newpage
\verbatimtabinput[6]{../../programa/include/ensambla.h}
\newpage
\end{codigo}


\section{Maquina virtual}
\begin{codigo}
\verbatimtabinput[6]{../../programa/mv/mv.cpp}
\newpage
\verbatimtabinput[6]{../../programa/mv/memfis.cpp}
\newpage
\verbatimtabinput[6]{../../programa/mv/mmu.cpp}
\newpage
\verbatimtabinput[6]{../../programa/mv/cpu.cpp}
\newpage
\verbatimtabinput[6]{../../programa/mv/teclado.cpp}
\newpage
\verbatimtabinput[6]{../../programa/mv/pantalla.cpp}
\newpage
\verbatimtabinput[6]{../../programa/mv/disco.cpp}
\newpage
\end{codigo}

\section{Nucleo}
\begin{codigo}
\verbatimtabinput[6]{../../programa/nucleo/nucleo.cpp}
\newpage
\verbatimtabinput[6]{../../programa/nucleo/memoria.cpp}
\newpage
\verbatimtabinput[6]{../../programa/nucleo/hilo.cpp}
\newpage
\verbatimtabinput[6]{../../programa/nucleo/proceso.cpp}
\newpage
\verbatimtabinput[6]{../../programa/nucleo/cola.cpp}
\newpage
\verbatimtabinput[6]{../../programa/nucleo/procesos.cpp}
\newpage
\verbatimtabinput[6]{../../programa/nucleo/es/controla.cpp}
\newpage
\verbatimtabinput[6]{../../programa/nucleo/es/ccar.cpp}
\newpage
\verbatimtabinput[6]{../../programa/nucleo/es/cteclado.cpp}
\newpage
\verbatimtabinput[6]{../../programa/nucleo/es/cpantall.cpp}
\newpage
\verbatimtabinput[6]{../../programa/nucleo/es/cbloque.cpp}
\newpage
\verbatimtabinput[6]{../../programa/nucleo/es/cdisco.cpp}
\newpage
\verbatimtabinput[6]{../../programa/nucleo/es/caracter.cpp}
\newpage
\verbatimtabinput[6]{../../programa/nucleo/es/bloque.cpp}
\newpage
\verbatimtabinput[6]{../../programa/nucleo/sf/sfv.cpp}
\newpage
\verbatimtabinput[6]{../../programa/nucleo/sf/flocal.cpp}
\newpage
\verbatimtabinput[6]{../../programa/nucleo/sf/eafitsf.cpp}
\newpage
\verbatimtabinput[6]{../../programa/nucleo/sf/efichero.cpp}
\newpage
\verbatimtabinput[6]{../../programa/nucleo/ejec/ejec99.cpp}
\newpage
\end{codigo}


\section{Entorno}
\begin{codigo}
\verbatimtabinput[6]{../../programa/entorno/entorno.cpp}
\newpage
\verbatimtabinput[6]{../../programa/entorno/ic.cpp}
\newpage
\verbatimtabinput[6]{../../programa/entorno/ensambla.cpp}
\newpage
\verbatimtabinput[6]{../../programa/eafitos2.cpp}
\newpage
\end{codigo}

\chapter{Pr�ctica: sem�foros}

\section{Planteamiento}
Eafitos es un sistema operativo multitarea, pero no dispone de ning�n
mecanismo de sincronizaci�n entre hilos.
Esto convierte la multitarea en algo bastante in�til.
El objetivo de esta pr�ctica es implementar un mecanismo de sincronizaci�n
con el cual sea f�cil controlar el acceso a secciones cr�ticas de c�digo.


\section{Soluci�n}
El mecanismo elegido para implementar son los sem�foros.

Vamos a implementar dos nuevas llamadas al sistema que recibir�n un solo
par�metro.
Ese par�metro ser� una direcci�n de memoria y actuar� a modo de cerradura.
Una llamada servir� para reservar una cerradura y la otra para liberarla.
Cuando un hilo solicite una cerradura, si no est� reservada se le
conceder� y seguir� ejecut�ndose, si est� reservada el hilo pasar� a suspendido
hasta que la cerradura que solicit� sea liberada.

Para implementar esto primero crearemos dos nuevas clases llamadas
\id{Semaforo} y \id{Semaforos} (encontraras su c�digo en la
Secci�n~\ref{Sec: Codigo Semaforo}.

Adem�s, habr� que hacer otras tres peque�as modificaciones:
\begin{itemize}
\item A�adir un atributo m�s a la clase \id{Nucleo}:
\begin{quote}
\begin{verbatim}
static Semaforos semaforos;
\end{verbatim}
\end{quote}

\item
Definir dos nuevas constantes en \emph{include/nucleo.h}:
\begin{quote}
\begin{verbatim}
#define RESERVAR_CERRADURA 4
#define LIBERAR_CERRADURA 5
\end{verbatim}
\end{quote}

\item
A�adir dos entradas en el bloque \id{switch} del m�todo \id{Nucleo::llamada:}
\begin{quote}
\begin{verbatim}
case RESERVAR_CERRADURA:
        param1 = cpu.desapilar();
        Nucleo::semaforos.reservar(param1);
        break;
case LIBERAR_CERRADURA:
        param1 = cpu.desapilar();
        Nucleo::semaforos.liberar(param1);
        break;
\end{verbatim}
\end{quote}
\end{itemize}

\section{C�digo nuevo}
\label{Sec: Codigo Semaforo}
\begin{codigo}
\verbatimtabinput{../../pracs/semaforo/semaforo.h}
\newpage
\verbatimtabinput{../../pracs/semaforo/semaforo.cpp}
\newpage
\end{codigo}

\section{Ejemplo}
El c�digo que viene a continuaci�n es un ejemplo de programa de Eafitos
que hace uso de los sem�foros.

\verbatimtabinput{../../pracs/semaforo/semaforo}

% Typeset by Matt Welsh (mdw@tc.cornell.edu)
% This file is covered by its own copyright...

\chapter{The GNU General Public License}
\label{GPL}

\bigskip
\centerline{{\bf GNU GENERAL PUBLIC LICENSE}}
\centerline{Version 2, June 1991}

Copyright (C) 1989, 1991 Free Software Foundation, Inc.
675 Mass Ave, Cambridge, MA 02139, USA
Everyone is permitted to copy and distribute verbatim copies
of this license document, but changing it is not allowed.

\section{Preamble}
%\centerline{{\sc Preamble}}

The licenses for most software are designed to take away your
freedom to share and change it.  By contrast, the GNU General Public
License is intended to guarantee your freedom to share and change free
software--to make sure the software is free for all its users.  This
General Public License applies to most of the Free Software
Foundation's software and to any other program whose authors commit to
using it.  (Some other Free Software Foundation software is covered by
the GNU Library General Public License instead.)  You can apply it to
your programs, too.

When we speak of free software, we are referring to freedom, not
price.  Our General Public Licenses are designed to make sure that you
have the freedom to distribute copies of free software (and charge for
this service if you wish), that you receive source code or can get it
if you want it, that you can change the software or use pieces of it
in new free programs; and that you know you can do these things.

To protect your rights, we need to make restrictions that forbid
anyone to deny you these rights or to ask you to surrender the rights.
These restrictions translate to certain responsibilities for you if you
distribute copies of the software, or if you modify it.

For example, if you distribute copies of such a program, whether
gratis or for a fee, you must give the recipients all the rights that
you have.  You must make sure that they, too, receive or can get the
source code.  And you must show them these terms so they know their
rights.

We protect your rights with two steps: (1) copyright the software, and
(2) offer you this license which gives you legal permission to copy,
distribute and/or modify the software.

Also, for each author's protection and ours, we want to make certain
that everyone understands that there is no warranty for this free
software.  If the software is modified by someone else and passed on, we
want its recipients to know that what they have is not the original, so
that any problems introduced by others will not reflect on the original
authors' reputations.

Finally, any free program is threatened constantly by software
patents.  We wish to avoid the danger that redistributors of a free
program will individually obtain patent licenses, in effect making the
program proprietary.  To prevent this, we have made it clear that any
patent must be licensed for everyone's free use or not licensed at all.

The precise terms and conditions for copying, distribution and
modification follow.
                    
%\centerline{{\sc GNU GENERAL PUBLIC LICENSE}}
%\centerline{{\sc TERMS AND CONDITIONS FOR COPYING, DISTRIBUTION AND 
%MODIFICATION}}
\section[Terms and Conditions]
{Terms and Conditions for Copying, Distribution, and Modification}

\begin{enumerate}
\item[0.] This License applies to any program or other work which contains
a notice placed by the copyright holder saying it may be distributed
under the terms of this General Public License.  The ``Program'', below,
refers to any such program or work, and a ``work based on the Program''
means either the Program or any derivative work under copyright law:
that is to say, a work containing the Program or a portion of it,
either verbatim or with modifications and/or translated into another
language.  (Hereinafter, translation is included without limitation in
the term ``modification''.)  Each licensee is addressed as ``you''.

Activities other than copying, distribution and modification are not
covered by this License; they are outside its scope.  The act of
running the Program is not restricted, and the output from the Program
is covered only if its contents constitute a work based on the
Program (independent of having been made by running the Program).
Whether that is true depends on what the Program does.

\item[1.] You may copy and distribute verbatim copies of the Program's
source code as you receive it, in any medium, provided that you
conspicuously and appropriately publish on each copy an appropriate
copyright notice and disclaimer of warranty; keep intact all the
notices that refer to this License and to the absence of any warranty;
and give any other recipients of the Program a copy of this License
along with the Program.

You may charge a fee for the physical act of transferring a copy, and
you may at your option offer warranty protection in exchange for a fee.

\item[2.] You may modify your copy or copies of the Program or any portion
of it, thus forming a work based on the Program, and copy and
distribute such modifications or work under the terms of Section 1
above, provided that you also meet all of these conditions:

\begin{enumerate}
\item[a.] You must cause the modified files to carry prominent notices
    stating that you changed the files and the date of any change.

\item[b.] You must cause any work that you distribute or publish, that in
    whole or in part contains or is derived from the Program or any
    part thereof, to be licensed as a whole at no charge to all third
    parties under the terms of this License.

\item[c.] If the modified program normally reads commands interactively
    when run, you must cause it, when started running for such
    interactive use in the most ordinary way, to print or display an
    announcement including an appropriate copyright notice and a
    notice that there is no warranty (or else, saying that you provide
    a warranty) and that users may redistribute the program under
    these conditions, and telling the user how to view a copy of this
    License.  (Exception: if the Program itself is interactive but
    does not normally print such an announcement, your work based on
    the Program is not required to print an announcement.)
\end{enumerate}

These requirements apply to the modified work as a whole.  If
identifiable sections of that work are not derived from the Program,
and can be reasonably considered independent and separate works in
themselves, then this License, and its terms, do not apply to those
sections when you distribute them as separate works.  But when you
distribute the same sections as part of a whole which is a work based
on the Program, the distribution of the whole must be on the terms of
this License, whose permissions for other licensees extend to the
entire whole, and thus to each and every part regardless of who wrote it.

Thus, it is not the intent of this section to claim rights or contest
your rights to work written entirely by you; rather, the intent is to
exercise the right to control the distribution of derivative or
collective works based on the Program.

In addition, mere aggregation of another work not based on the Program
with the Program (or with a work based on the Program) on a volume of
a storage or distribution medium does not bring the other work under
the scope of this License.

\item[3.] You may copy and distribute the Program (or a work based on it,
under Section 2) in object code or executable form under the terms of
Sections 1 and 2 above provided that you also do one of the following:

\begin{enumerate}
\item[a.] Accompany it with the complete corresponding machine-readable
    source code, which must be distributed under the terms of Sections
    1 and 2 above on a medium customarily used for software interchange; or,

\item[b.] Accompany it with a written offer, valid for at least three
    years, to give any third party, for a charge no more than your
    cost of physically performing source distribution, a complete
    machine-readable copy of the corresponding source code, to be
    distributed under the terms of Sections 1 and 2 above on a medium
    customarily used for software interchange; or,

\item[c.] Accompany it with the information you received as to the offer
    to distribute corresponding source code.  (This alternative is
    allowed only for noncommercial distribution and only if you
    received the program in object code or executable form with such
    an offer, in accord with Subsection b above.)
\end{enumerate}

The source code for a work means the preferred form of the work for
making modifications to it.  For an executable work, complete source
code means all the source code for all modules it contains, plus any
associated interface definition files, plus the scripts used to
control compilation and installation of the executable.  However, as a
special exception, the source code distributed need not include
anything that is normally distributed (in either source or binary
form) with the major components (compiler, kernel, and so on) of the
operating system on which the executable runs, unless that component
itself accompanies the executable.

If distribution of executable or object code is made by offering
access to copy from a designated place, then offering equivalent
access to copy the source code from the same place counts as
distribution of the source code, even though third parties are not
compelled to copy the source along with the object code.
  
\item[4.] You may not copy, modify, sublicense, or distribute the Program
except as expressly provided under this License.  Any attempt
otherwise to copy, modify, sublicense or distribute the Program is
void, and will automatically terminate your rights under this License.
However, parties who have received copies, or rights, from you under
this License will not have their licenses terminated so long as such
parties remain in full compliance.

\item[5.] You are not required to accept this License, since you have not
signed it.  However, nothing else grants you permission to modify or
distribute the Program or its derivative works.  These actions are
prohibited by law if you do not accept this License.  Therefore, by
modifying or distributing the Program (or any work based on the
Program), you indicate your acceptance of this License to do so, and
all its terms and conditions for copying, distributing or modifying
the Program or works based on it.

\item[6.] Each time you redistribute the Program (or any work based on the
Program), the recipient automatically receives a license from the
original licensor to copy, distribute or modify the Program subject to
these terms and conditions.  You may not impose any further
restrictions on the recipients' exercise of the rights granted herein.
You are not responsible for enforcing compliance by third parties to
this License.

\item[7.] If, as a consequence of a court judgment or allegation of patent
infringement or for any other reason (not limited to patent issues),
conditions are imposed on you (whether by court order, agreement or
otherwise) that contradict the conditions of this License, they do not
excuse you from the conditions of this License.  If you cannot
distribute so as to satisfy simultaneously your obligations under this
License and any other pertinent obligations, then as a consequence you
may not distribute the Program at all.  For example, if a patent
license would not permit royalty-free redistribution of the Program by
all those who receive copies directly or indirectly through you, then
the only way you could satisfy both it and this License would be to
refrain entirely from distribution of the Program.

If any portion of this section is held invalid or unenforceable under
any particular circumstance, the balance of the section is intended to
apply and the section as a whole is intended to apply in other
circumstances.

It is not the purpose of this section to induce you to infringe any
patents or other property right claims or to contest validity of any
such claims; this section has the sole purpose of protecting the
integrity of the free software distribution system, which is
implemented by public license practices.  Many people have made
generous contributions to the wide range of software distributed
through that system in reliance on consistent application of that
system; it is up to the author/donor to decide if he or she is willing
to distribute software through any other system and a licensee cannot
impose that choice.

This section is intended to make thoroughly clear what is believed to
be a consequence of the rest of this License.

\item[8.] If the distribution and/or use of the Program is restricted in
certain countries either by patents or by copyrighted interfaces, the
original copyright holder who places the Program under this License
may add an explicit geographical distribution limitation excluding
those countries, so that distribution is permitted only in or among
countries not thus excluded.  In such case, this License incorporates
the limitation as if written in the body of this License.

\item[9.] The Free Software Foundation may publish revised and/or new versions
of the General Public License from time to time.  Such new versions will
be similar in spirit to the present version, but may differ in detail to
address new problems or concerns.

Each version is given a distinguishing version number.  If the Program
specifies a version number of this License which applies to it and ``any
later version'', you have the option of following the terms and conditions
either of that version or of any later version published by the Free
Software Foundation.  If the Program does not specify a version number of
this License, you may choose any version ever published by the Free Software
Foundation.

\item[10.] If you wish to incorporate parts of the Program into other free
programs whose distribution conditions are different, write to the author
to ask for permission.  For software which is copyrighted by the Free
Software Foundation, write to the Free Software Foundation; we sometimes
make exceptions for this.  Our decision will be guided by the two goals
of preserving the free status of all derivatives of our free software and
of promoting the sharing and reuse of software generally.

\bigskip
\centerline{{\sc NO WARRANTY}}

\item[11.] BECAUSE THE PROGRAM IS LICENSED FREE OF CHARGE, THERE IS NO WARRANTY
FOR THE PROGRAM, TO THE EXTENT PERMITTED BY APPLICABLE LAW.  EXCEPT WHEN
OTHERWISE STATED IN WRITING THE COPYRIGHT HOLDERS AND/OR OTHER PARTIES
PROVIDE THE PROGRAM ``AS IS'' WITHOUT WARRANTY OF ANY KIND, EITHER EXPRESSED
OR IMPLIED, INCLUDING, BUT NOT LIMITED TO, THE IMPLIED WARRANTIES OF
MERCHANTABILITY AND FITNESS FOR A PARTICULAR PURPOSE.  THE ENTIRE RISK AS
TO THE QUALITY AND PERFORMANCE OF THE PROGRAM IS WITH YOU.  SHOULD THE
PROGRAM PROVE DEFECTIVE, YOU ASSUME THE COST OF ALL NECESSARY SERVICING,
REPAIR OR CORRECTION.

\item[12.] IN NO EVENT UNLESS REQUIRED BY APPLICABLE LAW OR AGREED TO IN WRITING
WILL ANY COPYRIGHT HOLDER, OR ANY OTHER PARTY WHO MAY MODIFY AND/OR
REDISTRIBUTE THE PROGRAM AS PERMITTED ABOVE, BE LIABLE TO YOU FOR DAMAGES,
INCLUDING ANY GENERAL, SPECIAL, INCIDENTAL OR CONSEQUENTIAL DAMAGES ARISING
OUT OF THE USE OR INABILITY TO USE THE PROGRAM (INCLUDING BUT NOT LIMITED
TO LOSS OF DATA OR DATA BEING RENDERED INACCURATE OR LOSSES SUSTAINED BY
YOU OR THIRD PARTIES OR A FAILURE OF THE PROGRAM TO OPERATE WITH ANY OTHER
PROGRAMS), EVEN IF SUCH HOLDER OR OTHER PARTY HAS BEEN ADVISED OF THE
POSSIBILITY OF SUCH DAMAGES.

\end{enumerate}
\centerline{{\sc END OF TERMS AND CONDITIONS}}

%\bigskip
%\centerline{{\sc Appendix: How to Apply These Terms to Your New Programs}}
\section[How to Apply These Terms]
{How to Apply These Terms to Your New Programs}

If you develop a new program, and you want it to be of the greatest
possible use to the public, the best way to achieve this is to make it
free software which everyone can redistribute and change under these terms.

To do so, attach the following notices to the program.  It is safest
to attach them to the start of each source file to most effectively
convey the exclusion of warranty; and each file should have at least
the ``copyright'' line and a pointer to where the full notice is found.

\begin{quote}
    \cparam{one line to give the program's name and a brief idea of 
    what it does.}
    Copyright \copyright 19yy  \cparam{name of author}

    This program is free software; you can redistribute it and/or modify
    it under the terms of the GNU General Public License as published by
    the Free Software Foundation; either version 2 of the License, or
    (at your option) any later version.

    This program is distributed in the hope that it will be useful,
    but WITHOUT ANY WARRANTY; without even the implied warranty of
    MERCHANTABILITY or FITNESS FOR A PARTICULAR PURPOSE.  See the
    GNU General Public License for more details.

    You should have received a copy of the GNU General Public License
    along with this program; if not, write to the Free Software
    Foundation, Inc., 675 Mass Ave, Cambridge, MA 02139, USA.
\end{quote}

Also add information on how to contact you by electronic and paper mail.

If the program is interactive, make it output a short notice like this
when it starts in an interactive mode:

\begin{tscreen}
    Gnomovision version 69, Copyright (C) 19yy name of author
    Gnomovision comes with ABSOLUTELY NO WARRANTY; for details type `show w'.
    This is free software, and you are welcome to redistribute it
    under certain conditions; type `show c' for details.
\end{tscreen}

The hypothetical commands `show w' and `show c' should show the appropriate
parts of the General Public License.  Of course, the commands you use may
be called something other than `show w' and `show c'; they could even be
mouse-clicks or menu items--whatever suits your program.

You should also get your employer (if you work as a programmer) or your
school, if any, to sign a ``copyright disclaimer'' for the program, if
necessary.  Here is a sample; alter the names:

\begin{quote}
  Yoyodyne, Inc., hereby disclaims all copyright interest in the program
  `Gnomovision' (which makes passes at compilers) written by James Hacker.

  \cparam{signature of Ty Coon}, 1 April 1989
  Ty Coon, President of Vice
\end{quote}

This General Public License does not permit incorporating your program into
proprietary programs.  If your program is a subroutine library, you may
consider it more useful to permit linking proprietary applications with the
library.  If this is what you want to do, use the GNU Library General
Public License instead of this License.



\chapter{GNU GPL en castellano}
\label{GPL en castellano}

\bigskip
\centerline{{\bf LICENCIA P�BLICA GENERAL DE GNU}}
\centerline{Versi�n 2, Junio de 1991
\footnote{Traduccida al catellano por Jes\'{u}s M. Gonz�lez, 
    jgb@gsyc.inf.uc3m.es.}}

Copyright (C) 1989, 1991 Free Software Foundation, Inc.
675 Mass Ave, Cambridge, MA 02139, EEUU
Se permite la copia y distribuci�n de copias literales
de este documento, pero no se permite su modificaci�n.


\section{Pre�mbulo}

Las licencias que cubren la mayor parte del software est�n dise�adas
para quitarle a usted la libertad de compartirlo y modificarlo. Por el
contrario, la Licencia P�blica General de GNU pretende garantizarle
la libertad de compartir y modificar software libre, para asegurar que el
software es libre para todos sus usuarios. Esta Licencia P�blica
General se aplica a la mayor parte del software del la Free Software
Foundation y a cualquier otro programa si sus autores se comprometen a
utilizarla. (Existe otro software de la Free Software Foundation que
est� cubierto por la Licencia P�blica General de GNU para Bibliotecas). 
Si quiere, tambi�n puede aplicarla a sus propios programas.

Cuando hablamos de software libre, estamos refiri�ndonos a libertad,
no a precio. Nuestras Licencias P�blicas Generales est�n dise�adas
para asegurarnos de que tenga la libertad de distribuir copias de
software libre (y cobrar por ese servicio si quiere), de que reciba
el c�digo fuente o que pueda conseguirlo si lo quiere, de que
pueda modificar el software o usar fragmentos de �l en nuevos
programas libres, y de que sepa que puede hacer todas estas
cosas.

Para proteger sus derechos necesitamos algunas restricciones que
prohiban a cualquiera negarle a usted estos derechos o pedirle que renuncie a
ellos. Estas restricciones se traducen en ciertas obligaciones que le
afectan si distribuye copias del software, o si lo modifica.

Por ejemplo, si distribuye copias de uno de estos programas, sea
gratuitamente, o a cambio de una contraprestaci�n, debe dar a los
receptores todos los derechos que tiene. Debe asegurarse de que
ellos tambi�n reciben, o pueden conseguir, el c�digo fuente. Y
debe mostrarles estas condiciones de forma que conozcan sus derechos.

Protegemos sus derechos con la combinaci�n de dos medidas:
\begin{enumerate}
\item Ponemos el software bajo copyright y
\item le ofrecemos esta licencia, que le da permiso legal para copiar,
  distribuir y/o modificar el software.
\end{enumerate}

Tambi�n, para la protecci�n de cada autor y la nuestra propia,
queremos asegurarnos de que todo el mundo comprende que no se
proporciona ninguna garant�a para este software libre. Si el software
se modifica por cualquiera y �ste a su vez lo distribuye, queremos
que sus receptores sepan que lo que tienen no es el original, de forma
que cualquier problema introducido por otros no afecte a la reputaci�n
de los autores originales.

Por �ltimo, cualquier programa libre est� constantemente amenazado
por patentes sobre el software. Queremos evitar el peligro de que los
redistribuidores de un programa libre obtengan 
patentes por su cuenta, convirtiendo de facto el programa en
propietario. Para evitar esto, hemos dejado claro que cualquier
patente debe ser pedida para el uso libre de cualquiera, o no ser
pedida.

Los t�rminos exactos y las condiciones para la copia, distribuci�n y
modificaci�n se exponen a continuaci�n.

\section[T�rminos y condiciones]{T�rminos y condiciones para la copia,
	distribuci�n y modificaci�n}

\begin{enumerate}
\item[0.] Esta Licencia se aplica a cualquier programa u otro tipo
de trabajo que contenga una nota colocada por el tenedor del copyright
diciendo que puede ser distribuido bajo los t�rminos de esta Licencia
P�blica General. En adelante, ``Programa'' se referir� a cualquier
programa o trabajo que cumpla esa condici�n y ``trabajo basado en el
Programa'' se referir� bien al Programa o a cualquier trabajo
derivado de �l seg�n la ley de copyright. Esto es, un trabajo que
contenga el programa o una proci�n de �l, bien en forma literal o
con modificaciones y/o traducido en otro lenguaje. Por lo tanto, la
traducci�n est� incluida sin limitaciones en el t�rmino
``modificaci�n''. Cada concesionario (licenciatario) ser� denominado
``usted''.

Cualquier otra actividad que no sea la copia, distribuci�n o
modificaci�n no est� cubierta por esta Licencia, est� fuera de su
�mbito. El acto de ejecutar el Programa no est� restringido, y los
resultados del Programa est�n cubiertos �nicamente si sus contenidos
constituyen un trabajo basado en el Programa, independientemente de
haberlo producido mediante la ejecuci�n del programa. El que esto se
cumpla, depende de lo que haga el programa.

\item[1.] Usted puede copiar y distribuir copias literales del c�digo
fuente del Programa, seg�n lo has recibido, en cualquier medio,
supuesto que de forma adecuada y bien visible publique en cada copia
un anuncio de copyright adecuado y un repudio de garant�a, mantenga
intactos todos los anuncios que se refieran a esta Licencia y a la
ausencia de garant�a, y proporcione a cualquier otro receptor del
programa una copia de esta Licencia junto con el Programa.

Puede cobrar un precio por el acto f�sico de transferir una copia, y
puede, seg�n su libre albedr�o, ofrecer garant�a a cambio de unos
honorarios.

\item[2.] Puede modificar su copia o copias del Programa o de
cualquier porci�n de �l, formando de esta manera un trabajo basado
en el Programa, y copiar y distribuir esa modificaci�n o trabajo bajo
los t�rminos del apartado 1, antedicho, supuesto que adem�s
cumpla las siguientes condiciones:

\begin{enumerate}
  \item[a.] Debe hacer que los ficheros modificados lleven anuncios
    prominentes indicando que los ha cambiado y la fecha de cualquier
    cambio.
  \item[b.] Debe hacer que cualquier trabajo que distribuya o
    publique y que en todo o en parte contenga o sea derivado del
    Programa o de cualquier parte de �l sea licenciada como un todo,
    sin carga alguna, a todas las terceras partes y bajo los t�rminos
    de esta Licencia.
  \item[c.] Si el programa modificado lee normalmente �rdenes
    interactivamente cuando es ejecutado, debe hacer que, cuando
    comience su ejecuci�n para ese uso interactivo de la forma m�s
    habitual, muestre o escriba un mensaje que incluya un anuncio de
    copyright y un anuncio de que no se ofrece ninguna garant�a (o
    por el contrario que s� se ofrece garant�a) y que los usuarios
    pueden redistribuir el programa bajo estas condiciones, e
    indicando al usuario c�mo ver una copia de esta
    licencia. (Excepci�n: si el propio programa es interactivo pero
    normalmente no muestra ese anuncio, no se requiere que su trabajo
    basado en el Programa muestre ning�n anuncio).
\end{enumerate}

Estos requisitos se aplican al trabajo modificado como un todo. Si
partes identificables de ese trabajo no son derivadas del Programa, y
pueden, razonablemente, ser consideradas trabajos independientes y
separados por ellos mismos, entonces esta Licencia y sus t�rminos no
se aplican a esas partes cuando sean distribuidas como trabajos
separados. Pero cuando distribuya esas mismas secciones como partes
de un todo que es un trabajo basado en el Programa, la distribuci�n
del todo debe ser seg�n los t�rminos de esta licencia, cuyos
permisos para otros licenciatarios se extienden al todo completo, y
por lo tanto a todas y cada una de sus partes, con independencia de
qui�n la escribi�.

Por lo tanto, no es la intenci�n de este apartado reclamar derechos o
desafiar sus derechos sobre trabajos escritos totalmente por usted
mismo. El intento es ejercer el derecho a controlar la distribuci�n
de trabajos derivados o colectivos basados en el Programa.

Adem�s, el simple hecho de reunir un trabajo no basado en el Programa
con el Programa (o con un trabajo basado en el Programa) en un volumen
de almacenamiento o en un medio de distribuci�n no hace que dicho
trabajo entre dentro del �mbito cubierto por esta Licencia.

\item[3.] Puede copiar y distribuir el Programa (o un trabajo
basado en �l, seg�n se especifica en el apartado 2, como c�digo
objeto o en formato ejecutable seg�n los t�rminos de los apartados 1
y 2, supuesto que adem�s cumpla una de las siguientes condiciones:

\begin{enumerate}
  \item[a.] Acompa�arlo con el c�digo fuente completo
    correspondiente, en formato electr�nico, que debe ser distribuido
    seg�n se especifica en los apartados 1 y 2 de esta Licencia en un
    medio habitualmente utilizado para el intercambio de programas, o
  \item[b.] Acompa�arlo con una oferta por escrito, v�lida
    durante al menos tres a�os, de proporcionar a cualquier tercera
    parte una copia completa en formato
    electr�nico del c�digo fuente correspondiente, a un coste no
    mayor que el de realizar f�sicamente la 
    distribuci�n del fuente, que ser�
    distribuido bajo las condiciones descritas en los apartados 1 y 2
    anteriores, en un medio habitualmente utilizado para el
    intercambio de programas, o
  \item[c.] Acompa�arlo con la informaci�n que recibiste
    ofreciendo distribuir el c�digo fuente correspondiente. (Esta
    opci�n se permite s�lo para distribuci�n no comercial y s�lo
    si usted recibi� el programa como c�digo objeto o en formato
    ejecutable con tal oferta, de acuerdo con el apartado b anterior).
\end{enumerate}

Por c�digo fuente de un trabajo se entiende la forma preferida del
trabajo cuando se le hacen modificaciones. Para un trabajo ejecutable,
se entiende por c�digo fuente completo todo el c�digo fuente para
todos los m�dulos que contiene, m�s cualquier fichero asociado de
definici�n de interfaces, m�s los guiones utilizados para controlar
la compilaci�n e instalaci�n del ejecutable. Como excepci�n
especial el c�digo fuente distribuido no necesita incluir nada que
sea distribuido normalmente (bien como fuente, bien en forma binaria)
con los componentes principales (compilador, kernel y similares) del
sistema operativo en el cual funciona el ejecutable, a no ser que el
propio componente acompa�e al ejecutable.

Si la distribuci�n del ejecutable o del c�digo objeto se hace
mediante la oferta acceso para copiarlo de un cierto lugar,
entonces se considera la oferta de acceso para copiar el c�digo
fuente del mismo lugar como distribuci�n del c�digo fuente, incluso
aunque terceras partes no est�n forzadas a copiar el fuente junto con
el c�digo objeto.

\item[4.] No puede copiar, modificar, sublicenciar o distribuir el
Programa excepto como prev� expresamente esta Licencia. Cualquier
intento de copiar, modificar sublicenciar o distribuir el Programa de
otra forma es inv�lida, y har� que cesen autom�ticamente los
derechos que te proporciona esta Licencia. En cualquier caso, las
partes que hayan recibido copias o derechos de usted bajo esta Licencia
no cesar�n en sus derechos mientras esas partes contin�en
cumpli�ndola.

\item[5.] No est� obligado a aceptar esta licencia, ya que no la
ha firmado. Sin embargo, no hay hada m�s que le proporcione permiso
para modificar o distribuir el Programa o sus trabajos derivados. Estas
acciones est�n prohibidas por la ley si no acepta esta Licencia. Por
lo tanto, si modifica o distribuye el Programa (o cualquier trabajo
basado en el Programa), est� indicando que acepta esta Licencia
para poder hacerlo, y todos sus t�rminos y condiciones para copiar,
distribuir o modificar el Programa o trabajos basados en �l.

\item[6.] Cada vez que redistribuya el Programa (o cualquier
trabajo basado en el Programa), el receptor recibe autom�ticamente
una licencia del licenciatario original para copiar, distribuir o
modificar el Programa, de forma sujeta a estos t�rminos y
condiciones. No puede imponer al receptor ninguna restricci�n m�s sobre el
ejercicio de los derechos aqu� garantizados. No es usted responsable de
hacer cumplir esta licencia por terceras partes.

\item[7.] Si como consecuencia de una resoluci�n judicial o
de una alegaci�n de infracci�n de patente o por cualquier otra raz�n (no
limitada a asuntos relacionados con patentes) se le imponen
condiciones (ya sea por mandato judicial, por acuerdo o por cualquier
otra causa) que contradigan las condiciones de esta Licencia, ello no
le exime de cumplir las condiciones de esta Licencia. Si no puede
realizar distribuciones de forma que se satisfagan simult�neamente
sus obligaciones bajo esta licencia y cualquier otra
obligaci�n pertinente entonces, como consecuencia, no puede
distribuir el Programa de ninguna forma. Por ejemplo, si una patente
no permite la redistribuci�n libre de derechos de autor del Programa
por parte de todos aquellos que reciban copias directa o
indirectamente a trav�s de usted, entonces la �nica forma en que
podr�a satisfacer tanto esa condici�n como esta Licencia ser�a
evitar completamente la distribuci�n del Programa.

Si cualquier porci�n de este apartado se considera inv�lida o
imposible de cumplir bajo cualquier circunstancia particular ha de
cumplirse el resto y la secci�n por entero ha de cumplirse en
cualquier otra circunstancia.

No es el prop�sito de este apartado inducirle a infringir ninguna
reivindicaci�n de patente ni de ning�n otro derecho de propiedad o
impugnar la validez de ninguna de dichas reivindicaciones. Este
apartado tiene el �nico prop�sito de proteger la integridad del
sistema de distribuci�n de software libre, que se realiza mediante
pr�cticas de licencia p�blica. Mucha gente ha hecho contribuciones
generosas a la gran variedad de software distribuido mediante ese
sistema con la confianza de que el sistema se aplicar�
consistentemente. Ser� el autor/donante quien decida si quiere
distribuir software mediante cualquier otro sistema y una licencia no
puede imponer esa elecci�n.

Este apartado pretende dejar completamente claro lo que se cree que es
una consecuencia del resto de esta Licencia.

\item[8.] Si la distribuci�n y/o uso de el Programa est�
restringida en ciertos pa�ses, bien por patentes o por interfaces bajo
copyright, el tenedor del copyright que coloca este Programa bajo esta
Licencia puede a�adir una limitaci�n expl�cita de distribuci�n
geogr�fica excluyendo esos pa�ses, de forma que la distribuci�n se
permita s�lo en o entre los pa�ses no excluidos de esta manera. En
ese caso, esta Licencia incorporar� la limitaci�n como si estuviese
escrita en el cuerpo de esta Licencia.

\item[9.] La Free Software Foundation puede publicar
versiones revisadas y/o nuevas de la Licencia P�blica General de
tiempo en tiempo. Dichas nuevas versiones ser�n similares en
esp�ritu a la presente versi�n, pero pueden ser diferentes en
detalles para considerar nuevos problemas o situaciones.

Cada versi�n recibe un n�mero de versi�n que la distingue de
otras. Si el Programa especifica un n�mero de versi�n de esta
Licencia que se refiere a ella y a ``cualquier versi�n posterior'',
tienes la opci�n de seguir los t�rminos y condiciones, bien de esa
versi�n, bien de cualquier versi�n posterior publicada por la Free
Software Foundation. Si el Programa no especifica un n�mero de
versi�n de esta Licencia, puedes escoger cualquier versi�n publicada
por la Free Software Foundation.

\item[10.] Si quiere incorporar partes del Programa en otros
programas libres cuyas condiciones de distribuci�n son diferentes,
escribe al autor para pedirle permiso. Si el software tiene copyright
de la Free Software Foundation, escribe a la Free Software
Foundation: algunas veces hacemos excepciones en estos casos. Nuestra
decisi�n estar� guiada por el doble objetivo de de preservar la
libertad de todos los derivados de nuestro software libre y promover
el que se comparta y reutilice el software en general.

\bigskip
\centerline{{\sc AUSENCIA DE GARANT�A}}

\item[11.] COMO EL PROGRAMA SE LICENCIA LIBRE DE CARGAS, NO SE
OFRECE NINGUNA GARANT�A SOBRE EL PROGRAMA, EN TODA LA EXTENSI�N
PERMITIDA POR LA LEGISLACI�N APLICABLE. EXCEPTO CUANDO SE INDIQUE DE
OTRA FORMA POR ESCRITO, LOS TENEDORES DEL COPYRIGHT Y/U OTRAS PARTES
PROPORCIONAN EL PROGRAMA ``TAL CUAL'', SIN GARANT�A DE NINGUNA CLASE,
BIEN EXPRESA O IMPL�CITA, CON INCLUSI�N, PERO SIN LIMITACI�N A LAS
GARANT�AS MERCANTILES IMPL�CITAS O A LA CONVENIENCIA PARA UN
PROP�SITO PARTICULAR. CUALQUIER RIESGO REFERENTE A LA CALIDAD Y
PRESTACIONES DEL PROGRAMA ES ASUMIDO POR USTED. SI SE PROBASE QUE EL
PROGRAMA ES DEFECTUOSO, ASUME EL COSTE DE CUALQUIER SERVICIO,
REPARACI�N O CORRECCI�N.

\item[12.] EN NING�N CASO, SALVO QUE LO REQUIERA LA LEGISLACI�N
APLICABLE O HAYA SIDO ACORDADO POR ESCRITO, NING�N TENEDOR DEL
COPYRIGHT NI NINGUNA OTRA PARTE QUE MODIFIQUE Y/O REDISTRIBUYA El
PROGRAMA SEG�N SE PERMITE EN ESTA LICENCIA SER� RESPONSABLE ANTE USTED
POR DA�OS, INCLUYENDO CUALQUIER DA�O GENERAl, ESPECIAL, INCIDENTAL O
RESULTANTE PRODUCIDO POR EL USO O LA IMPOSIBILIDAD DE USO DEL PROGRAMA
(CON INCLUSI�N, PERO SIN LIMITACI�N A LA P�RDIDA DE DATOS O A LA
GENERACI�N INCORRECTA DE DATOS O A P�RDIDAS SUFRIDAS POR USTED O POR
TERCERAS PARTES O A UN FALLO DEL PROGRAMA AL FUNCIONAR EN COMBINACI�N
CON CUALQUIER OTRO PROGRAMA), INCLUSO SI DICHO TENEDOR U OTRA PARTE HA
SIDO ADVERTIDO DE LA POSIBILIDAD DE DICHOS DA�OS.

\end{enumerate}
\centerline{{\sc FIN DE T�RMINOS Y CONDICIONES}}

\section[C�mo aplicar estos t�rmins]{C�mo aplicar estos t�rminos a sus
	nuevos programas}

Si usted desarrolla un nuevo Programa, y quiere que sea del mayor uso
posible para el p�blico en general, la mejor forma de conseguirlo es
convirti�ndolo en software libre que cualquiera pueda redistribuir y
cambiar bajo estos t�rminos. 

Para hacerlo, a�ada los siguientes anuncios al programa. Lo m�s
seguro es a�adirlos al principio de cada fichero fuente para
transmitir lo m�s efectivamente posible la ausencia de
garant�a. Adem�s cada fichero deber�a tener al menos la l�nea de
``copyright'' y un indicador a d�nde puede encontrarse el anuncio
completo.

\begin{quote}
	\cparam{una l�nea para indicar el nombre del programa y una
	r�pida idea de qu� hace.}
	Copyright (C) 19aa  \cparam{nombre del autor}

	Este programa es software libre. Puede redistribuirlo y/o
	modificarlo bajo los t�rminos de la Licencia P�blica General
	de GNU seg�n es publicada por la Free Software Foundation, bien
	de la versi�n 2 de dicha Licencia o bien (seg�n su elecci�n)
	de cualquier versi�n posterior.

	Este programa se distribuye con la esperanza de que sea �til,
	pero SIN NINGUNA GARANT�A, incluso sin la garant�a MERCANTIL
	impl�cita o sin garantizar la CONVENIENCIA PARA UN PROP�SITO
	PARTICULAR. V�ase la Licencia P�blica General de GNU para m�s
	detalles.

	Deber�a haber recibido una copia de la Licencia P�blica General
	junto con este programa. Si no ha sido as�, escriba a la Free
	Software Foundation, Inc., en 675 Mass Ave, Cambridge, MA 02139,
	EEUU.
\end{quote}

A�ada tambi�n informaci�n sobre c�mo contactar con usted mediante
correo electr�nico y postal.

Si el programa es interactivo, haga que muestre un peque�o anuncio
como el siguiente, cuando comienza a funcionar en modo interactivo:

\begin{tscreen}
  Gnomovision versi�n 69, Copyright (C) 19aa nombre del autor

  Gnomovision no ofrece ABSOLUTAMENTE NINGUNA GARANT�A.
  Para m�s detalles escriba ``show w''.

  Esto es software libre, y se le invita a redistribuirlo bajo
  ciertas condiciones. Escriba ``show c'' para m�s detalles.
\end{tscreen}

Los comandos hipot�ticos ``show w'' y ``show c'' deber�an mostrar
las partes adecuadas de la Licencia P�blica General. Por supuesto,
los comandos que use pueden llamarse de cualquier otra
manera. Podr�an incluso ser pulsaciones del rat�n o elementos de un
men� (lo que sea apropiado para su programa).

Tambi�n deber�as conseguir que su empleador (si trabaja como
programador) o tu Universidad (si es el caso) firme un ``renuncia de
copyright'' para el programa, si es necesario. A continuaci�n se
ofrece un ejemplo, altere los nombres seg�n sea conveniente:

\begin{quote}
 Yoyodyne, Inc. mediante este documento renuncia a cualquier inter�s
 de derechos de copyright con respecto al programa Gnomovision (que hace
 pasadas a compiladores) escrito por Pepe Programador.

 \cparam{firma de Pepito Grillo}, 20 de diciembre de 1996
 Pepito Grillo, Presidente de Asuntillos Varios.
\end{quote}

Esta Licencia P�blica General no permite que incluya sus programas
en programas propietarios. Si su programa es una biblioteca de
subrutinas, puede considerar m�s �til el permitir el enlazado de
aplicaciones propietarias con la biblioteca. Si este es el caso, use
la Licencia P�blica General de GNU para Bibliotecas en lugar de esta
Licencia.

\end{appendix}


\backmatter

\end{document}
