\chapter{Pr�ctica: sem�foros}

\section{Planteamiento}
Eafitos es un sistema operativo multitarea, pero no dispone de ning�n
mecanismo de sincronizaci�n entre hilos.
Esto convierte la multitarea en algo bastante in�til.
El objetivo de esta pr�ctica es implementar un mecanismo de sincronizaci�n
con el cual sea f�cil controlar el acceso a secciones cr�ticas de c�digo.


\section{Soluci�n}
El mecanismo elegido para implementar son los sem�foros.

Vamos a implementar dos nuevas llamadas al sistema que recibir�n un solo
par�metro.
Ese par�metro ser� una direcci�n de memoria y actuar� a modo de cerradura.
Una llamada servir� para reservar una cerradura y la otra para liberarla.
Cuando un hilo solicite una cerradura, si no est� reservada se le
conceder� y seguir� ejecut�ndose, si est� reservada el hilo pasar� a suspendido
hasta que la cerradura que solicit� sea liberada.

Para implementar esto primero crearemos dos nuevas clases llamadas
\id{Semaforo} y \id{Semaforos} (encontraras su c�digo en la
Secci�n~\ref{Sec: Codigo Semaforo}.

Adem�s, habr� que hacer otras tres peque�as modificaciones:
\begin{itemize}
\item A�adir un atributo m�s a la clase \id{Nucleo}:
\begin{quote}
\begin{verbatim}
static Semaforos semaforos;
\end{verbatim}
\end{quote}

\item
Definir dos nuevas constantes en \emph{include/nucleo.h}:
\begin{quote}
\begin{verbatim}
#define RESERVAR_CERRADURA 4
#define LIBERAR_CERRADURA 5
\end{verbatim}
\end{quote}

\item
A�adir dos entradas en el bloque \id{switch} del m�todo \id{Nucleo::llamada:}
\begin{quote}
\begin{verbatim}
case RESERVAR_CERRADURA:
        param1 = cpu.desapilar();
        Nucleo::semaforos.reservar(param1);
        break;
case LIBERAR_CERRADURA:
        param1 = cpu.desapilar();
        Nucleo::semaforos.liberar(param1);
        break;
\end{verbatim}
\end{quote}
\end{itemize}

\section{C�digo nuevo}
\label{Sec: Codigo Semaforo}
\begin{codigo}
\verbatimtabinput{../../pracs/semaforo/semaforo.h}
\newpage
\verbatimtabinput{../../pracs/semaforo/semaforo.cpp}
\newpage
\end{codigo}

\section{Ejemplo}
El c�digo que viene a continuaci�n es un ejemplo de programa de Eafitos
que hace uso de los sem�foros.

\verbatimtabinput{../../pracs/semaforo/semaforo}
