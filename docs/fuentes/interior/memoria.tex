\chapter{El n�cleo: gesti�n de memoria}

La parte del n�cleo responsable de la gesti�n de memoria est� representada
por la clase \id{ManejadorMemoria.}
\marginnote{Ver include/memoria.h}
El manejador de memoria se apoya en la unidad de gesti�n de memoria de la
m�quina virtual para realizar su tarea, de hecho la mayor parte del trabajo
lo realiza la \emph{MMU}.


\section{Reservar y liberar memoria}
Esta es la tarea principal del manejador de memoria.
Para realizarla disponemos de tres m�todos, \id{asignar,} \id{reservar} y
\id{liberar.}
Estos m�todos lo �nico que hacen es llamar a sus equivalentes en la unidad
de gesti�n de memoria.
El primero, \id{asignar,} reserva la memoria solicitada y devuelve el
\emph{selector de segmento} reservado; el m�todo \id{reservar} incrementa el
uso del segmento especificado; finalmente, \id{liberar} reduce el uso del
segmento especificado y, si ya no se utiliza (su uso es cero), libera al
segmento y a la memoria asociada.

\section{Copiar desde y hacia espacio de usuario}
En los sistemas operativos reales suelen existir unas funciones que copian
datos de memoria de usuario a memoria del n�cleo y viceversa.
En Eafitos tambi�n, solo que la memoria del n�cleo no reside en la memoria
de la m�quina virtual, sino directamente en la memoria del sistema operativo
anfitri�n.

Estos m�todos son \id{aUsuario} y \id{deUsuario}, los cuales copian,
respectivamente, datos de memoria del n�cleo a memoria de usuario y viceversa.
\marginnote{Ver nucleo/memoria.cpp}
Para se ello utilizan los m�todos \id{CPU::escribirByte} y \id{CPU::leerByte}.

\section{Modificar el modelo de memoria}
Modificar el modelo de memoria implica no solo modificar el manejador de
memoria, sino sobre todo modificar la unidad de gesti�n de memoria en la
m�quina virtual, ya que la mayor complejidad reside en la \emph{MMU}.
De hecho, dependiendo de los cambios que se quieran realizar, es probable que
no sea necesario modificar el manejador de memoria en nada.
El modelo de memoria no est� pensado especialmente para que sea f�cilmente
modificable, pero algunas cambios sencillos no deber�an ser dif�ciles de
hacer, ya que los mecanismos b�sicos de paginaci�n y segmentaci�n est�n
implementados.
